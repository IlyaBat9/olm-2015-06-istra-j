% $date: 2015-06-03
% $timetable:
%   g9s: {}

\section*{Задачи к спецкурсу <<Геометрия коник>>}

% $authors:
% - Фёдор Нилов

\begin{problems}

\item
Найдите геометрическое место центров окружностей, касающихся двух данных
окружностей.

\item
Проверить, что конструкция невидимого объекта, предложенного на лекции,
является корректной.

\item\emph{Оптическое свойство гиперболы.}
Дана гипербола с фокусами $F_1$ и $F_2$.
$M$~--- произвольная точка на гиперболе.
Докажите, что касательная в точке~$M$ к гиперболе является биссектрисой
угла $F_1 M F_2$.

\item\emph{Изогональное свойство параболы.}
Дана парабола с фокусом~$F$.
Из точки~$X$ вне параболы проведены касательные $XP$ и $XQ$ к этой параболе.
Докажите, что угол между прямыми $XP$ и $XF$ равен углу между $XQ$ и осью
параболы.

\item
Хорда эллипса $PQ$ проходит через его фокус~$F$.
Касательные в точках $P$ и $Q$ пересекаются в точке~$X$.
Докажите, что прямые $XF$ и $PQ$ перпендикулярны.

\item
Дан эллипс с фокусом~$F$.
Из точки~$X$ вне эллипса проведены касательные $XP$ и $XQ$
($P$ и $Q$~--- точки касания).
Докажите, что углы $PFX$ и $QFX$ равны.

\item
Пусть вокруг коники с фокусом~$F$ описан $2n$-угольник, стороны которого
окрашены попеременно в черный и белый цвета.
Докажите, что сумма углов, под которыми из $F$ видны черные стороны
многоугольника, равна $180^{\circ}$.

\item
Найдите геометрическое место проекций фокуса параболы на ее касательные.

\item
Докажите, что луч света при отражении от коники касается фиксированной
софокусной коники.

\item
Найдите геометрическое место точек, из которых данный эллипс виден под прямым
углом.

\end{problems}
  
