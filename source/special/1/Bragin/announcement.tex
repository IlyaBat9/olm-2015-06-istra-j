% $date: 2--4 июня 2015
% $timetable:
%   g9s: {}

\section*{Перестановки}

% $authors:
% - Владимир Брагин

% $style[-announcement]:
% - .[announcement]

В~первой волне спецкурсов В.\,А.\,Брагин расскажет про перестановки.
Про то, что~же это за~объекты, как с~ними работать.
Как понимание их~устройства помогает ответить на~вопрос, сколько различных
комбинаций Кубика Рубика можно получить, почему нельзя поменять 15 и~14
в~пятнашках, сколько раз надо специфичным образом помешать  колоду, чтобы
ничего не~изменилось.
Заденем и~самую прямую связь перестановок с~движениями.
Пусть сложные слова в~анонсе никого не~пугают.
От~слушателей требуется лишь математическое воображение и~знание азов
комбинаторики.

\bigskip

%\[
%    \text{\fbox{\(\displaystyle
%    \begin{matrix}
%         1 &  2 &  3 &  4 \\
%         5 &  6 &  7 &  8 \\
%         9 & 10 & 11 & 12 \\
%        13 & 15 & 14 & \phantom{16}
%    \end{matrix}\)}}
%\qquad
%    \text{\fbox{\(\displaystyle
%    \begin{matrix}
%         1 &  2 &  3 &  4 \\
%         5 &  6 &  7 &  8 \\
%         9 & 10 & 11 & 12 \\
%        13 & 14 & 15 & \phantom{16}
%    \end{matrix}\)}}
%\]

\begin{center}
\jeolmfigure[width=0.2\linewidth]{right}
\quad
\jeolmfigure[width=0.2\linewidth]{wrong}
\end{center}

