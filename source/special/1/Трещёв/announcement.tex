% $date: первая серия
% $timetable:
%   g9s: {}

\section*{Многоугольники на решетках}

% $authors:
% - Виктор Трещёв

% $style[-announcement]:
% - .[announcement]

Все мы~привыкли решать задачки на~бумаге в~клеточку.
Оказывается, сам клетчатый листок таит в~себе множество интересных фактов.

Рассмотрим набор параллельных прямых на~плоскости, таких, что расстояние между
любыми двумя соседними прямыми одно и~то~же.
И~еще один такой~же набор, непараллельный первому.
Множество точек пересечения этих двух наборов называется \emph{решеткой},
а~сами эти точки~--- \emph{узлами решетки}.
Нас будут интересовать многоугольники, все вершины которых попали в~узлы
решетки.

Приходите на~спецкурс, и~вы~узнаете, как быстро искать площади таких
многоугольников и~какие значения эти площади могут принимать, какие правильные
и~полуправильные многоугольники можно изобразить на~различных решетках,
познакомитесь с~пространственными решетками и~порешаете интересные задачи.

\bigskip

\begin{center}
    \jeolmfigure[width=0.3\linewidth]{picture}
\end{center}

