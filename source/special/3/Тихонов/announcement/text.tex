Практически все листочки с~задачами, которые вам выдаются на~этих сборах,
подготовлены в~различных вариациях системы {\LaTeX} (произносится л\'{а}тех).
На~занятиях вы~научитесь набирать математические тексты в~этой системе.

Требования к~слушателям:
\begin{itemize}
\item
\textbf{компьютер в~пользовании во~время занятий спецкурса,}
с~современным браузером (Google Chrome, например).
\item
наличие собственного e-mail адреса.
\item
смирение с~тем фактом, что компьютер выполнит любую данную ему инструкцию,
и~результат не~обязательно совпадет с~вашими ожиданиями.
\end{itemize}

Нужно будет зарегистрироваться на~сайте www.sharelatex.com.
Можно сделать это заранее.

Зачет по~спецкурсу будет ставится тем, кто наберет решения нескольких задач
из~своего курса
(как минимум по~одной из~геометрии, алгебры и~комбинаторики;
как минимум с~одной геометрической картинкой;
количество задач зависит от~самих задач).

Если у~вас нет компьютера, вы~можете попрововать уговорить кого-нибудь дать его
вам на~время занятий.
В~крайнем случае, можно объединиться с~кем-нибудь в~пару
(но~тогда и~требования к~зачету соответственно возрастут).

