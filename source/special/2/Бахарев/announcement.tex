% $date: 6--8 июня 2015
% $timetable:
%   g9s: {}

\section*{Биномиальные коэффициенты и числа Стирлинга}

% $authors:
% - Фёдор Львович Бахарев

% $style[-announcement]:
% - .[announcement]

Символ $\binom{n}{k}$~--- это биномиальный коэффициент, своим названием
обязанный волшебной биномиальной теореме~--- важному соотношению, которое мы,
конечно, на~спецкурсе разберем.
А~читается этот символ как <<выбор $k$ из~$n$>>.
Подобное заклинание проистекает из~его комбинаторной интерпретации~--- это
число способов выбора $k$-элементного подмножества из~$n$-элементного
множества.
Скажем, два элемента из~множества $\{ 1, 2, 3, 4 \}$ можно выбрать шестью
способами, поэтому $\binom{4}{2} = 6$.
Биномиальные коэффициенты обладают массой замечательных свойств, некоторые
из~которых мы~обсудим в~первой половине курса. 
 
Однако, у~биномиальных коэффициентов есть два очень близких родственника,
которые почему-то куда менее известны школьникам.
Речь идет о~числах Стирлинга.
Почему два родственника?
Да потому что числа Стирлинга бывают двух видов, а~точнее, двух родов~---
первого и~второго.
Числа Стрилинга тоже можно интерпретировать очень красивым комбинаторным
образом.
А~роднит их~с~биномиальными коэффициентами наличие аналогичной биномиальной
теоремы, только для альтернативным образом определенных степеней чисел.
Ну и, кроме всего прочего, числа Стирлинга, как и~подобает родственникам,
имеют очень много свойств, схожих со~свойствами биномиальных коэффициентов.

