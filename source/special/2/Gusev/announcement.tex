% $date: 6--8 июня 2015
% $timetable:
%   g9s: {}

\section*{Разборчивая невеста}

% $authors:
% - Антон Сергеевич Гусев

% $style[-announcement]:
% - .[announcement]

\begin{quote}
Пусть в~некотором царстве, в~некотором государстве принцесса решила, что
ей~пора найти себе жениха.
Созвали царевичей и~королевичей со~всего света, и~явилось 1000 претендентов.
Про любых двух когда-либо увиденных принцесса может сказать, кто из~них лучше.
При этом царевичи, как говорят математики, образуют упорядоченное множество,
т.~е. если Иван Царевич лучше Василия Царевича, а~Василий Царевич лучше
Фёдора Царевича, то~Иван Царевич лучше Фёдора Царевича.
Претенденты входят к~принцессе по~очереди, по~одному, причем их~порядок
определен случайным образом, т.~е. вероятность появления какого-то царевича
первым, или пятисотым, или тысячным совершенно одинакова.
Принцесса, разумеется, умея их~сравнивать, может сказать, что, например,
вошедший тридцатым является десятым по~качеству, т.~е. девять из~предыдущих
были лучше, а~остальные — хуже, и~т.~д.
Цель принцессы — получить самого хорошего жениха, т.~е. даже второй
ее~не~устраивает.
На~каждом шаге, т.~е. после встречи с~каждым из~царевичей, она решает, берет~ли
она его в~мужья.
Если берет, то~на~этом смотр претендентов заканчивается, они все разъезжаются
по~домам.
Если~же принцесса ему отказывает, то~царевич, будучи отвергнутым, тут~же
уезжает домой, потому что все царевичи и~королевичи~--- люди гордые.
Показ претендентов на~замужество при этом продолжается.
Если в~конце концов принцесса не~получает лучшего, то~считается, что она
проиграла, выходить замуж вообще не~будет, а~уйдет в~монастырь.
Как действовать принцессе, чтобы с~наибольшей вероятностью
получить лучшего жениха?
\end{quote}

В~ходе нашего спецкурса мы~разберемся, как решать эту вполне естественную
задачу.
А~также поговорим о~некоторых ее~обобщениях.
Приходите, должно быть интересно~;)

