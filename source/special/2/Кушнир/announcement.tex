% $date: вторая серия
% $timetable:
%   g9s: {}

\section*{Комбинаторная теория игр}

% $authors:
% - Андрей Юрьевич Кушнир

% $style[-announcement]:
% - .[announcement]

Игра <<Ним>> имеет очень простые правила:
есть несколько кучек камней, за~один ход разрешается взять любое количество
камней из~любой кучки;
проигрывает тот, кто не~может сделать ход.
На~спецкурсе мы~докажем, что \emph{любая} симметричная игра с~конечным числом
состояний и~конечным числом потенциальных ходов \emph{эквивалентна} <<Ним>>.
В~процессе доказательства мы~будем использовать необычную технику:
введем на~множестве игр разные алгебраические структуры, т.~е. научимся
сравнивать и~складывать (и~может быть даже перемножать) игры.
Неожиданно окажется, что эти структуры интересны сами по~себе и~тесно связаны
с~другими разделами математики.
Для понимания курса предварительных знаний не~требуется.

