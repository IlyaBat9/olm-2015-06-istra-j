% $date: 2015-06-01
% $timetable:
%   g9r3:
%     2015-06-01:
%       3:

\section*{Путешествия по графам}

% $authors:
% - Фёдор Нилов

\begin{problems}

\item
В~некотором государстве каждый город соединен с~каждым дорогой.
Сумасшедший король хочет ввести на~дорогах одностороннее движение так, чтобы
выехав из~любого города, в~него нельзя было вернуться.
Можно~ли так сделать?

\item
В~стране Мара расположено несколько замков.
Из~каждого замка ведут три дороги.
Из~какого-то замка выехал рыцарь.
Странствуя по~дорогам, он~из~каждого замка, стоящего на~его пути, поворачивает
либо направо, либо налево по~отношению к~дороге, по~которой приехал.
Рыцарь никогда не~сворачивает в~ту сторону, в~которую он~свернул перед этим.
Доказать, что когда-нибудь он~вернется в~исходный замок.

\item
Докажите, что связный граф, имеющий не~более двух нечетных вершин, можно
нарисовать, не~отрывая карандаша от~бумаги и~проводя каждое ребро ровно один
раз.

\item
Докажите, что связный граф, имеющий не~более $2 n$ нечетных вершин, можно
нарисовать, отрывая карандаш от~бумаги $(n - 1)$ раз и~проводя каждое ребро
ровно один раз.

\item
Любознательный турист хочет прогуляться по~улицам Старого города от~вокзала
(точка~$A$ на~плане) до~своего отеля (точка~$B$).
Турист хочет, чтобы его маршрут был как можно длиннее, но~дважды оказываться
на~одном и~том~же перекрестке ему неинтересно, и~он~так не~делает.
Нарисуйте на~плане самый длинный возможный маршрут и~докажите, что более
длинного нет.

\begin{center}
    \jeolmfigure[width=0.2\linewidth]{city}
\end{center}

\item
Поселок построен в~виде квадрата $3$~квартала на $3$~квартала (кварталы~---
квадраты со~стороной~$b$, всего $9$~кварталов).
Какой наименьший путь должен пройти асфальтоукладчик, чтобы заасфальтировать
все улицы, если он~начинает и~кончает свой путь в~угловой точке $A$?
(Стороны квадрата~--- тоже улицы).

\item
Можно~ли $n$ раз рассадить $2 n + 1$ человек за~круглым столом, чтобы никакие
двое не~сидели рядом более одного раза, если
\\
\subproblem $n = 5$;
\quad
\subproblem $n = 4$;
\\
\subproblem $n$~--- произвольное натуральное число?

\item
В~стране несколько городов, соединенных дорогами с~односторонним и~двусторонним
движением.
Известно, что из~каждого города в~любой другой можно проехать ровно одним
путем, не~проходящим два раза через один и~тот~же город.
Докажите, что страну можно разделить на~три губернии так, чтобы ни~одна дорога
не~соединяла два города из~одной губернии.

\item
В~стране $n$ городов.
Между каждыми двумя из~них проложена либо автомобильная, либо железная дорога.
Турист хочет объехать страну, побывав в~каждом городе ровно один раз,
и~вернуться в~город, с~которого он~начинал путешествие.
Докажите, что турист может выбрать город, с~которого он~начнет путешествие,
и~маршрут так, что ему придется поменять вид транспорта не~более одного раза.

\end{problems}

