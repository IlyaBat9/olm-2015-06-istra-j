% $date: 2015-06-02
% $timetable:
%   g9r3:
%     2015-06-02:
%       1:

\section*{Графы-2}

% $authors:
% - Фёдор Нилов

\begin{problems}

\item
В~графе каждая вершина~--- синяя или зеленая.
При этом каждая синяя вершина связана с~$5$ синими и~$10$ зелеными, а~каждая
зеленая~--- с~$9$ синими и~$6$ зелеными.
Каких вершин больше~--- синих или зеленых?

\item
Докажите, что среди любых шести людей есть либо трое попарно знакомых, либо
трое попарно незнакомых.
Верно~ли это утверждение для пяти людей?

\item
Можно~ли нарисовать на~плоскости $9$~отрезков так, чтобы каждый пересекался
ровно с~тремя другими?

\item
В~какое наибольшее число цветов можно покрасить ребра куба так, чтобы для любых
двух цветов нашлись два ребра, окрашенные в~эти цвета, имеющие общую вершину?

\item
Верно~ли, что для любого $n$ существует граф, все ребра которого являются
единичными отрезками, а~степень каждой вершины равна $n$?

\item
Приведите пример графа, все ребра которого являются единичными отрезками,
хроматическое число которого равно $4$ (хроматическим числом называется
наименьшее число цветов, в~которое можно покрасить вершины данного граф так,
чтобы соседние вершины были покрашены в~разные цвета).

\item
Несколько Совершенно Секретных Объектов соединены подземной железной дорогой
таким образом,
что каждый Объект напрямую соединен не~более чем с~тремя другими, и~от~каждого
Объекта можно добраться под землей до~любого другого, сделав не~более одной
пересадки.
Каково максимальное число Совершенно Секретных Объектов?

\item
Как соединить $50$~городов наименьшим числом авиалиний так, чтобы из~каждого
города можно было попасть в~любой,
сделав не~более двух пересадок?

\item
В~стране $n$~городов.
Между каждыми двумя городами установлено воздушное сообщение одной из~двух
авиакомпаний.
Докажите, из~этих двух авиакомпаний хотя~бы одна такова, что что из~любого
города можно попасть в~любой другой рейсами только этой авиакомпании.

\end{problems}

