% $date: 2015-06-03
% $timetable:
%   g9r3:
%     2015-06-03:
%       2:

\section*{Классическая комбинаторика}

% $authors:
% - Фёдор Нилов

\begin{problems}

\item
У~одного школьника есть 6 книг по~математике, а~у~другого~--- 8. 
Сколькими способами они могут обменять три книги одного на~три книги другого?

\item
Человек имеет 6 друзей и~в~течение 5 дней приглашает к~себе в~гости каких-то
троих из~них так, чтобы компания ни~разу не~повторялась.
Сколькими способами он~может это сделать?

\item
Сколькими способами можно расставить 12 белых и~12 черных шашек на~черных полях
шахматной доски?

\item
Сколькими способами можно составить комиссию из~3 человек, выбирая ее~членов
из~4 супружеских пар, но~так, чтобы члены одной семьи не~входили в~комиссию
одновременно?

\item
Сколько существует шестизначных чисел, у~которых по~три четных и~нечетных
цифры?

\item
Фабрика игрушек выпускает проволочные кубики, в~вершинах которых расположены
маленькие разноцветные шарики.
По~ГОСТу в~каждом кубике должны быть использованы шарики всех восьми цветов
(белого и~семи цветов радуги).
Сколько разных моделей кубиков может выпускать фабрика?

\item
Имеется 20 человек~--- 10 юношей и~10 девушек.
Сколько существует способов составить компанию, в~которой было~бы одинаковое
число юношей и~девушек?

\item
Сколькими способами натуральное число $n$ можно представить в~виде суммы
\\
\subproblem $k$ натуральных слагаемых?
\\
\subproblem $k$ неотрицательных целых слагаемых?
\\
(Представления, отличающиеся порядком слагаемых, считаются различными.)

\end{problems}

