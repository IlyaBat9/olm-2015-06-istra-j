% $date: 2015-06-07
% $timetable:
%   g9r3:
%     2015-06-07:
%       1:

\section*{Геометрическая комбинаторика}

% $authors:
% - Фёдор Нилов

\begin{problems}

\item
Замостите шахматную доску фигурками тетрамино в~форме буквы <<\textsf{T}>>.

\item
Как разрезать треугольник на~несколько треугольников так, чтобы никакие два
из~треугольников разбиения не~имели целой общей стороны?

\item
Каждую грань кубика разбили на~четыре равных квадрата и~раскрасили эти квадраты
в~три цвета так, чтобы квадраты, имеющие общую сторону, были покрашены в~разные
цвета.
Докажите, что в~каждый цвет покрашено по~8 квадратиков.

\item
Можно~ли покрасить 4 вершины куба в~красный цвет и~4 другие~--- в~синий так,
чтобы плоскость, проходящая через любые три точки одного цвета, содержала точку
другого цвета?

\item
Можно~ли нарисовать на~плоскости шесть точек и~так соединить
их~непересекающимися отрезками, что каждая точка будет соединена ровно
с~четырьмя другими?

\item
Прямая раскрашена в~два цвета.
Докажите, что на~ней найдутся три точки $A$, $B$ и~$C$, окрашенные в~один цвет
такие, что точка~$B$ является серединой отрезка~$AC$.

\item
Отметьте несколько точек и~несколько прямых так, чтобы на~каждой прямой лежало
ровно три отмеченные точки и~через каждую точку проходило ровно три отмеченные
прямые.

\item
Можно~ли на~плоскости разместить конечное число парабол так, чтобы
их~внутренние области покрыли всю плоскость?

\item
В~клетчатом квадрате $64 \times 64$ вырезали одну из~клеток.
Докажите, что оставшуюся часть квадрата можно разрезать на~уголки из~трех
клеток.

\item
На~плоскости даны несколько точек.
Известно, что для любых двух из~них на~прямой, проходящей через них, лежит еще 
одна данная точка.
Докажите, что все точки данного набора лежат на~одной прямой.

\end{problems}

