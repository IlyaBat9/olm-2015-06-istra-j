% $date: 2015-06-07
% $timetable:
%   g8r1:
%     2015-06-07:
%       1:

\section*{Моноварианты}

% $authors:
% - Глеб Погудин

\begin{problems}

\item
На~плоскости в~вершинах треугольника сидят кузнечики.
Каждую минуту один из~кузнечиков, находящийся в~точке~$A$, перепрыгивает через
другого кузнечика, находящегося в~точке~$B$, в~точку~$A_1$, причем так, чтобы
$2 AB = A_1 B$.
Смогут~ли кузнечики через несколько ходов оказаться в~начальном положении?

\item
В~каждой из~$n$~стран правит либо партия правых, либо партия левых.
Каждый год в~одной из~стран, $A$, может поменяться власть.
Это может произойти в~том случае, если в~большинстве граничащих со~страной~$A$
стран правит не~та партия, которая правит в~стране~$A$.
Докажите, что смены правительств не~могут продолжаться бесконечно.

\item
На~плоскости отмечено $200$ точек.
Каждая из~точек соединена отрезком ровно с~одной другой.
За~один раз Юлик может стереть пересекающиеся отрезки $AB$ и~$CD$ и~провести
вместо них $AC$ и~$BD$.
Докажите, что он~не~сможет промышлять этим вечно.

\item
На~бесконечном листе клетчатой бумаги несколько клеточек покрашено в~черный
цвет.
Каждую секунду клетка окрашивается в~тот цвет, которого больше всего среди
ее~соседей справа, слева и~сверху.
Докажите, что однажды вся плоскость станет белой.

\item
На~доске написано несколько чисел.
Каждую минуту Юлик вместо двух из~них пишет их~НОД и~НОК.
Докажите, что начиная с~некоторого момента ничего не~будет меняться.

\item
По~кругу стоит $101$~мудрец.
Каждый из~них либо считает, что Земля вращается вокруг Юпитера, либо считает,
что Юпитер вращается вокруг Земли.
Один раз в~минуту все мудрецы одновременно оглашают свои мнения.
Сразу после этого каждый мудрец, оба соседа которого думают иначе, чем он,
меняет свое мнение, а~остальные~--- не~меняют.
Докажите, что через некоторое время мнения перестанут меняться.

\item
Любитель Чисел каждое утро выписывает на~доску новое натуральное число, причем
так, чтобы новое число не~могло быть представлено в~виде суммы нескольких
(возможно, с~повторениями) предыдущих.
Сможет~ли он~заниматься этим вечно?

\item
\emph{Задача отозвана.}

\item
На~экране компьютера сгенерирована некоторая конечная последовательность нулей
и~единиц.
С~ней можно производить следующую операцию: набор цифр <<01>> заменять на~набор
цифр <<1000>>.
Может~ли такой процесс замен продолжаться бесконечно или когда-нибудь
он~обязательно прекратится?
\emph{(Указание: это по~духу похоже на~задачу с~шарами с~олимпиады.)}

\end{problems}

