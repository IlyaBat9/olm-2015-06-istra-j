% $date: 2015-06-07
% $timetable:
%   g8r2:
%     2015-06-07:
%       2:

\section*{Моноварианты}

% $authors:
% - Глеб Погудин

\begin{problems}

\item
В~стране несколько городов, попарные расстояния между которыми различны.
Путешественник отправился из~города~$A$ в~самый удаленный от~него город~$B$,
оттуда~--- в~самый удаленный от~него город~$C$, и~т.~д.
Докажите, что если $C$ не~совпадает с~$A$, то~путешественник никогда не~вернется
в~$A$.

\item
В~каждой из~$n$~стран правит либо партия правых, либо партия левых.
Каждый год в~одной из~стран, $A$, может поменяться власть.
Это может произойти в~том случае, если в~большинстве граничащих со~страной~$A$
стран правит не~та партия, которая правит в~стране~$A$.
Докажите, что смены правительств не~могут продолжаться бесконечно.

\item
На~плоскости отмечено $200$ точек.
Каждая из~точек соединена отрезком ровно с~одной другой.
За~один раз Юлик может стереть пересекающиеся отрезки $AB$ и~$CD$ и~провести
вместо них $AC$ и~$BD$.
Докажите, что он~не~сможет промышлять этим вечно.

\item
По~кругу стоят натуральные числа.
Каждую минуту между каждыми двумя соседними числами записывают их~НОД, после
чего исходные числа стираются.
Докажите, что однажды все числа станут равны.

\item
В~одной школе в~11Е и~11Ж классах учится по~$25$~человек.
Каждую день один человек из~двух классов может перейти в~другой при условии,
что в~своем классе у~него друзей меньше, чем в~другом.
Докажите, что однажды переходы прекратятся, причем произойдет это раньше, чем
через $1000$ дней.

\item
На~бесконечном листе клетчатой бумаги несколько клеточек закрашено в~черный
цвет.
Каждую секунду клетка окрашивается в~тот цвет, которого больше всего среди
ее~соседей справа, слева и~сверху.
Докажите, что однажды вся плоскость станет белой.

\item
На~доске написано несколько чисел.
Каждую минуту Юлик вместо двух из~них пишет их~НОД и~НОК.
Докажите, что начиная с~некоторого момента ничего не~будет меняться.

\item
По~кругу стоит $101$~мудрец.
Каждый из~них либо считает, что Земля вращается вокруг Юпитера, либо считает,
что Юпитер вращается вокруг Земли.
Один раз в~минуту все мудрецы одновременно оглашают свои мнения.
Сразу после этого каждый мудрец, оба соседа которого думают иначе, чем он,
меняет свое мнение, а~остальные~--- не~меняют.
Докажите, что через некоторое время мнения перестанут меняться.

\end{problems}

