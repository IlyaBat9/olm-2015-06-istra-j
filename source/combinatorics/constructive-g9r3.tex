% $date: 2015-06-06
% $timetable:
%   g9r3:
%     2015-06-06:
%       1:

\section*{Конструкции}

% $authors:
% - Фёдор Нилов

\begin{problems}

\item
На~доске записаны числа $1$, $2^1$, $2^2$, $2^3$, $2^4$, $2^5$.
Разрешается стереть любые два числа и~вместо них записать их~разность~---
неотрицательное число.
Может~ли на~доске в~результате нескольких таких операций остаться только
число~$15$?

\item
Кое-кто в~классе смотрит футбол, кое-кто~--- мультики, но~нет таких, кто
не~смотрит ни~то, ни~другое.
У~любителей мультиков средний балл по~математике меньше $4$, у~любителей
футбола~--- тоже меньше $4$.
Может~ли средний балл всего класса по~математике быть больше $4$?

\item
Аня, Боря и~Вася составляли слова из~заданных букв.
Все составили разное число слов: больше всех~--- Аня, меньше всех~--- Вася.
Затем ребята просуммировали очки за~свои слова.
Если слово есть у~двух игроков, за~него дается $1$~очко,
у~одного игрока~--- $2$~очка,
а слова, общие у~всех трех игроков, вычеркиваются.
Могло~ли так случиться, что больше всех очков набрал Вася, а~меньше всех~---
Аня?

\item
В~$10$ одинаковых кувшинов было разлито молоко~--- не~обязательно поровну,
но~каждый оказался заполнен не~более, чем на~$10\%$.
За~одну операцию можно выбрать кувшин и~отлить из~него любую часть поровну
в~остальные кувшины.
Докажите, что не~более чем за~$10$ таких операций можно добиться, чтобы во~всех
кувшинах молока стало поровну.

\item
Гриб называется плохим, если в~нем не~менее $10$~червей.
В~лукошке $90$ плохих и~$10$ хороших грибов.
Могут~ли все грибы стать хорошими после того, как некоторые черви переползут
из~плохих грибов в~хорошие?

\item
Том Сойер взялся покрасить очень длинный забор, соблюдая условие: любые две
доски, между которыми ровно две, ровно три или ровно пять досок, должны быть
окрашены в~разные цвета.
Какое наименьшее количество красок потребуется Тому для этой работы?

\item
В~вершинах $33$-угольника записали в~некотором порядке целые числа от~$1$ до~$33$.
Затем на~каждой стороне написали сумму чисел в~ее концах.
Могут~ли на~сторонах оказаться $33$ последовательных целых числа (в~каком-нибудь
порядке)?

\item
На~плоскости даны два равных многоугольника $F$ и~$F'$.
Известно, что все вершины многоугольника $F$ принадлежат $F'$
(могут лежать внутри него или на~границе).
Верно~ли, что все вершины этих многоугольников совпадают?

\item
Можно~ли записать в~строку $20$ чисел так, чтобы сумма любых трех
последовательных чисел была положительна, а~сумма всех $20$ чисел была
отрицательна?

\item
Можно~ли расставить в~вершинах куба натуральные числа так, чтобы в~каждой паре
чисел, связанных ребром, одно из~них делилось на~другое, а~во~всех других парах
такого не~было?

\end{problems}

