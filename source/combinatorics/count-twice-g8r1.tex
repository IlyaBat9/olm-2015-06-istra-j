% $date: 2015-06-09
% $timetable:
%   g8r1:
%     2015-06-09:
%       1:

\section*{Подсчет двумя способами}

% $authors:
% - Погудин Глеб

\begin{enumerate}

\item
$10$~друзей послали друг другу праздничные открытки, так что каждый послал
$5$~открыток.
Докажите, что найдутся двое, которые послали открытки друг другу.

\item
Можно~ли в~клетки квадрата $10 \times 10$ поставить некоторое количество
звездочек так, чтобы в~каждом квадрате $2 \times 2$ было ровно две звездочки,
а~в~каждом прямоугольнике $3 \times 1$~--- ровно одна звездочка?

\item
Докажите, что максимальное количество сторон выпуклого многоугольника, стороны
которого лежат на~диагоналях данного выпуклого $100$-угольника, не~больше
$100$.

\item
По~кругу расставлены цифры $1, 2, 3,\ldots, 9$ в~произвольном порядке.
Каждые три цифры, стоящие подряд по~часовой стрелке, образуют трехзначное
число.
Докажите, что сумма этих чисел не~зависит от~порядка цифр.

\item
По~кругу расставлены красные и~синие числа.
Каждое красное число равно сумме соседних чисел, а~каждое синее~--- полусумме
соседних чисел.
Докажите, что сумма красных чисел равна нулю.

\item
В~прямоугольной таблице, составленной из~положительных чисел, произведение
суммы чисел любого столбца на~сумму чисел любой строки равно числу, стоящему
на их~пересечении.
Доказать, что сумма всех чисел в~таблице равна единице.

\item
Дано натуральное число~$n$.
Рассматриваются такие тройки различных натуральных чисел $(a, b, c)$, что
$a + b + c = n$.
Возьмем наибольшую возможную такую систему троек, что никакие две тройки
системы не~имеют общих элементов.
Докажите, что в~ней не~больше $2 n / 9$ элементов.

\item
В~компании из~$n$~человек у~каждых двух двух людей не~меньше $m$ общих
знакомых, и~у~каждого не~больше $d$~знакомых.
Докажите, что
\[
    \frac{m (n - 1)}{2}
\leq
    \frac{d (d - 1)}{2}
\;.\]

\item
В~школе учатся $2015$ мальчиков и~$2015$ девочек.
Каждый школьник посещает не~более $100$ кружков.
Известно, что для любых двух школьников разного пола найдется кружок, который
посещают они оба.
Докажите, что есть кружок, который посещает хотя~бы $11$~мальчиков и~хотя~бы
$11$~девочек.

\end{enumerate}

