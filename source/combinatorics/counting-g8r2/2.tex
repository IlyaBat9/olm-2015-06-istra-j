% $date: 2015-06-10
% $timetable:
%   g8r2:
%     2015-06-10:
%       2:

\section*{Ц. из N, пока!}

% $authors:
% - Погудин Глеб

Через $C_{n}^{k}$ обозначается количество способов выбрать $k$~предметов
из~$n$.
Верна следующая формула:
\(
    C_{n}^{k} = \frac{n!}{k! (n - k)!}
\).

\begin{problems}

\item
Какое из~чисел $C_{500}^{100}$ и~$C_{500}^{200}$ больше и~почему?

\item
Сколькими способами можно выставить баллы за~ЕГЭ (от~$0$ до~$100$) так, чтобы
у~Кати было баллов больше, чем у~Сары, а~у~Сары~--- больше, чем у~Пети?

\item
В~тесте на~каждый из $20$~вопросов имеется $4$~варианта ответа.
Талантливый мальчик Петя Торт хочет ответить наобум, но~так, чтобы ответов
каждого типа было одинаковое количество.
Сколькими способами он~может это сделать?

\item
Сколько существует слов из $16$ букв~\textsf{А} и $7$ букв~\textsf{Б} таких,
чтобы подряд всегда шло четное число букв~\textsf{А}?
В~частности, \textsf{ААБББААББ} можно, а~\textsf{АААБББАББ}~--- нельзя.

\item
В~классе $10$~хулиганов и~$15$~интеллигентов.
Сколькими способами можно выстроить их~в~шеренгу так, чтобы хулиганы не~стояли
рядом?

\item
Глеб Александрович хочет выдать некоторым из $12$~школьников в~группе
по~конфете, а~некоторым шоколадку.
Сколькими способами он~может это сделать, чтобы оба ништяка получили ровно пять
человек?

\item
\subproblem
Сколькими способами можно выбрать из $19$~предметов больше половины?
\\
\subproblem А из~$20$?

\item
Сколько клетчатых прямоугольников содержит доска $8 \times 8$?

\item
В~выпуклом $n$-угольнике проведены все диагонали, и~никакие три из~них
не~пересекаются в~одной точке.
Сколько получилось точек пересечения?

\end{problems}

