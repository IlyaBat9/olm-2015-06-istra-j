% $date: 2015-06-09
% $timetable:
%   g8r2:
%     2015-06-09:
%       2:

\section*{Классическая комбинаторика}

% $authors:
% - Погудин Глеб

\begin{problems}

\item
У~Остапа Бендера есть десять поддельных паспортов.
В~целях конспирации очередному милиционеру он~показывает не~тот паспорт,
который показывал прошлому, и~не~тот, который позапрошлому.
Сколькими способами он~может пообщаться с~десятью стражами порядка?

\item
Сколькими способами может выстроиться в~очередь класс из $20$~человек при
условии, что двоечник Петя стоит в~очереди раньше отличницы Кати?

\item
Сколько существует шестизначных чисел, в~записи которых \emph{найдется}
хотя~бы одна четная цифра?

\item
Сколькими способами $24$ человека могут распределиться в~$3$ очереди по~$8$,
если Петя, Катя и~Сара должны стоять в~разных очередях?

\item
Сколькими способами можно расставить в~очередь $20$ детей, чтобы между Петей
и~Катей стояло ровно $4$ человека?

\item
Сколькими способами в~классе из~$20$ человек выбрать две непересекающиеся
группы?

\item
Сколькими способами можно расставить $20$ детям в~классе баллы за~ЕГЭ
(от~$0$ до~$100$), чтобы \emph{нашелся} ученик, набравший больше $90$ баллов?

\item
Сколькими способами можно выставить от~$0$ до~$100$ баллов за~ЕГЭ отличнице
Кате и~двоечнику Пете, чтобы у~Кати было баллов больше, чем у~Пети и
не~меньше $50$?

\item
Сколькими способами можно выстроить класс из~$20$ человек в~хоровод, чтобы Петя
был на~одинаковом расстоянии от~Кати и~Сары?

\item
А~если человек $21$?

\item
Сколькими способами можно выстроить класс из~$20$ человек в~очередь, чтобы Петя
находился на~одного человека ближе к~Кате, чем к~Саре?

\item
Сколькими способами можно разрезать палку длины $2014$ на~$3$ части так, чтобы
потом из~этих частей можно было~бы сложить невырожденный треугольник?

\end{problems}

