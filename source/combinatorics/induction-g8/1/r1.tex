% $date: 2015-06-02
% $timetable:
%   g8r1:
%     2015-06-02:
%       2:

\section*{Индукция-1}

% $authors:
% - Глеб Погудин

\begin{problems}

\item
В~квадрате $1024 \times 1024$ вырезали одну клетку.
Докажите, что оставшееся можно разрезать на~уголки из~трех клеток.

\item\emph{Блинная сортировка.}
На~полке стоит $55$ томов собрания сочинений В.\,И.\,Ленина.
За~раз разрешается взять несколько подряд идущих томов и~переставить
их~в~обратном порядке.
Докажите, что такими операциями можно расставить тома по~порядку.

\item
Петя умеет на~любом отрезке отмечать точки, которые делят этот отрезок пополам
или в~отношении $n : (n + 1)$, где $n$~--- любое натуральное число.
Петя утверждает, что этого достаточно, чтобы разделить отрезок на~любое
количество одинаковых частей.
Прав~ли он?

\item
Плоскость разбита прямыми на~области.
Докажите, что области можно покрасить в~два цвета так, чтобы граничащие области
были разного цвета.

\item
Докажите, что любое целое число единственным образом представимо в~виде суммы
некоторых из~чисел $1, -2, 4, \ldots, (-2)^n, \ldots$.
Попробуйте придумать хотя~бы два способа сделать переход индукции.

\item
Из~чисел от~$1$ до~$2n$ выбрано $n + 1$ число.
Докажите, что среди выбранных чисел найдутся два, одно из~которых делится на~другое.

\item
Несколько человек не~знакомы между собой.
Нужно так познакомить друг с~другом некоторых из~них, чтобы ни~у~каких трех
людей не~оказалось одинакового числа знакомых.
Докажите, что это всегда можно сделать.

\item
Для того, чтобы проехать полный круг, машине требуется $100$ литров бензина.
Эти $100$ литров распределены по~бензоколонкам вдоль трассы.
Докажите, что есть точка, начав с~которой с~пустым баком, машина проедет весь
круг.

\item
По~кругу стоит $2015$ чисел $\pm 1$.
Известно, что количество $-1$ не~больше $671$.
Докажите, что найдется число такое, что все частичные суммы, начинающиеся с~него (что вправо, что влево), положительны.

\item
Можно~ли отметить на~плоскости несколько точек так, чтобы на~расстоянии $1$ от~каждой отмеченной точки находилось ровно $10$ отмеченных?

\end{problems}

