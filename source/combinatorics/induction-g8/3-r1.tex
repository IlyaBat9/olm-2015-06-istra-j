% $date: 2015-06-11
% $timetable:
%   g8r1:
%     2015-06-11:
%       2:

\section*{Добавка к индукции}

% $authors:
% - Глеб Погудин

\begin{problems}

\item
Обозначим через $a_n$ количество способов замостить прямоугольник $2 \times n$
доминошками.
Докажите, что $a_n = f_n$.

\item
Через $s(n)$ будем обозначать количество разбиений числа $n$ в~упорядоченную
сумму степеней двойки (например, $s(4) = 6$).
Найдите наименьшее $n$ большее $2015$ такое, что $f(n)$ нечетно.
\emph{(Указание: проведите численные эксперименты)}

\item
Докажите $f_{n}^{2} + f_{n + 1}^{2} = f_{2n + 1}$,
где $f_n$~--- $n$-ое число Фибоначчи.

\item
Определим числа $K_n$:
$K_0 = 1$
и~\(
    K_{n + 1}
=
    1 + \min (2 K_{[n / 2]}, 3 K_{[n / 3]})
\).
Докажите, что $K_n \geq n$.

\item
Назовем натуральное число \emph{ровным}, если в~его записи все цифры одинаковы
(например: $4$, $111$, $999999$).
Докажите, что любое $n$-значное число можно представить как сумму не~более чем 
$n + 1$ ровных чисел.

\item
Положительные числа $x_1, \ldots, x_n$ удовлетворяют
$x_1 + x_2 + \ldots + x_n = 1 / 2$.
Докажите, что:
\[
    \frac
        {(1 - x_1) \cdot (1 - x_2) \cdot \ldots \cdot (1 - x_n)}
        {(1 + x_1) \cdot (1 + x_2) \cdot \ldots \cdot (1 + x_n)}
\geq
    \frac{1}{3}
\;.\]

\item
Докажите неравенство
\[
    \frac{1}{2} \cdot
    \frac{3}{4} \cdot
    \ldots \cdot
    \frac{2 n - 1}{2 n}
<
    \frac{1}{\sqrt{3 n}}
\;.\]

\end{problems}

