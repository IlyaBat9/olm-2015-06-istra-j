% $date: 2015-06-03
% $timetable:
%   g8r1:
%     2015-06-03:
%       1:

\section*{Индукция-2: индукция и алгоритмы}

% $authors:
% - Глеб Погудин

\begin{problems}

\item
В~компании из~$k$~человек, где $k > 3$, у~каждого появилась новость, известная
ему одному.
За~один телефонный разговор двое сообщают друг другу все известные им~новости.
Докажите, что за~$(2 k - 4)$ разговора все они могут узнать все новости.

\item
Даны $n$~карточек.
На~обеих сторонах каждой карточки написано по~одному из~чисел
$1, 2, \ldots, n$, причем так, что каждое число встречается на~всех
$n$~карточках ровно два раза.
Доказать, что карточки можно разложить на~столе так, что сверху окажутся все
числа $1, 2, \ldots, n$.

\item
На~полке стоит $55$~томов собрания сочинений В.\,И.\,Ленина.
За~раз разрешается сделать любое из~двух действий:
\\
\textbf{(1)} поменять местами первые два тома на~полке;
\\
\textbf{(2)} переставить последний том в~начало.
\\
Докажите, что таким образом можно упорядочить все творческое наследие дедушки
Ленина.

\item
В~шеренгу выстроилось $n$~солдат.
За~одну операцию можно поменять любых двух солдат местами, после чего делается
групповая фотография.
Докажите, что за~$n!$ операций можно получить все возможные фотографии.

\item
В~$n$~мензурок налиты $n$~разных жидкостей, кроме того, имеется одна пустая
мензурка.
Можно~ли за~конечное число операций составить равномерные смеси в~каждой
мензурке, то~есть сделать так, чтобы в~каждой мензурке было равно $1 / n$
от~начального количества каждой жидкости, и~при этом одна мензурка была~бы
пустой?
(Мензурки одинаковые, но~количества жидкостей в~них могут быть разными;
предполагается, что можно отмерять любой объем жидкости.)

\item
Есть четное число комнат, в~каждой по~три лампочки.
Лампочки разбиты на~пары (в~паре могут быть лампочки из~разных комнат).
На~каждую пару по~одному выключателю, он~при нажатии меняет состояние обеих
лампочек в~паре на~противоположное.
Докажите, что вне зависимости от~того, какие лампочки горели в~начале, можно
сделать так, чтобы в~каждой комнате хотя~бы одна лампочка горела и~хотя~бы одна
не~горела.

\item
На~доске написаны в~каком-то порядке числа от~$1$ до~$2015$.
Каждую минуту происходит следующее: первые $k$~чисел переписываются в~обратном
порядке, где $k$~--- первое число в~строке.
Докажите, что рано или поздно на~первом месте будет число~$1$.

\end{problems}

