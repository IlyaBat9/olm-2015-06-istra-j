% $date: 2015-06-03
% $timetable:
%   g8r2:
%     2015-06-03:
%       2:

\section*{Индукция-2: индукция и алгоритмы}

% $authors:
% - Глеб Погудин

\begin{problems}

\item
На~столе стоят $1024$ стаканов с~водой.
Разрешается взять любые два стакана и~уравнять в~них количества воды, перелив
часть воды из~одного стакана в~другой.
Докажите, что с~помощью таких операций можно добиться того, чтобы во~всех
стаканах было поровну воды.

\item
Полк солдат подошел к~реке.
По~реке катались на~лодке два мальчика.
Лодка выдерживает одного солдата или двух мальчиков.
Как всем солдатам переправиться на~другой берег и~вернуть лодку мальчикам?

\item
В~компании из~$k$~человек, где $k > 3$, у~каждого появилась новость, известная
ему одному.
За~один телефонный разговор двое сообщают друг другу все известные им~новости.
Докажите, что за~$(2 k - 4)$ разговора все они могут узнать все новости.

\item
Даны $n$~карточек.
На~обеих сторонах каждой карточки написано по~одному из~чисел
$1, 2, \ldots, n$, причем так, что каждое число встречается на~всех
$n$~карточках ровно два раза.
Доказать, что карточки можно разложить на~столе так, что сверху окажутся все
числа $1, 2, \ldots, n$.

\item
Есть компания из $2 n$~человек.
Докажите, что их~можно перезнакомить так, чтобы у~двоих было ровно по~одному
знакомому, у~двоих~--- ровно по~$2$, \ldots, у~двоих~--- ровно по~$n$.

\item
На~полке стоит $55$~томов собрания сочинений В.\,И.\,Ленина.
За~раз разрешается сделать любое из~двух действий:
\\
\textbf{(1)} поменять местами первые два тома на~полке;
\\
\textbf{(2)} переставить последний том в~начало.
\\
Докажите, что таким образом можно упорядочить все творческое наследие дедушки
Ленина.

\item
В~шеренгу выстроилось $n$~солдат.
За~одну операцию можно поменять любых двух солдат местами, после чего делается
групповая фотография.
Докажите, что за~$n!$ операций можно получить все возможные фотографии.

\end{problems}

