% $date: 2015-06-04
% $timetable:
%   g8r2:
%     2015-06-04:
%       2:

\section*{Зацикливания}

% $authors:
% - Погудин Глеб

\begin{problems}

\item
Кубик Рубика вывели из~исходного состояния некоторой последовательностью
поворотов граней.
Докажите, что если повторять эту последовательность поворотов достаточно долго,
то~кубик в~конце концов вернется в~исходное состояние.

\item
Рассмотрим последовательность цифр, где каждая следующая равна последней цифре
произведения двух предыдущих.
Докажите, что
\\
\subproblem эта последовательность периодична;
\\
\subproblem ее~период не~больше $26$;
\\
\subproblem ее~период не~больше $17$.

\item
Докажите, что последовательность остатков при делении $16^n + 26^n$ на~$2015$
периодична.

\item
Последовательность $a_0, a_1, a_2,\ldots$ образована по~закону:
$a_0 = a_1 = 1$ и~ $a_{n + 1} = a_n \cdot a_{n - 1} + 1$.
Делится~ли число $a_{2015}$ на~$4$?

\item
Числа Фибоначчи задаются формулами
$f_1 = f_2 = 1$ и~$f_{n + 1} = f_n + f_{n - 1}$.
Докажите, что последовательность последних цифр этих чисел периодична, причем
не~имеет предпериода.

\item
Последовательность $a_n$ определена формулами:
$a_1 = a$ и~$a_{n + 1} = a_n + \ldots + a_1$.
Периодична~ли последовательность остатков от~деления $a_n$ на~$27$?

\item
Докажите, что цифры после запятой у~числа $a / n$ образуют периодическую
последовательность.

\item
Докажите, что для любого $m$ найдется число Фибоначчи, делящееся на~$m$.

\end{problems}

