% $date: 2015-06-04
% $timetable:
%   g8r1:
%     2015-06-04:
%       1:

\section*{Зацикливания}

% $authors:
% - Погудин Глеб

\begin{problems}

\item
Рассмотрим последовательность цифр, где каждая следующая равна последней цифре
произведения двух предыдущих.
Докажите, что
\\
\subproblem эта последовательность периодична;
\\
\subproblem ее~период не~больше $26$;
\\
\subproblem ее~период не~больше $17$.

\item
Числа Фибоначчи задаются формулами
$f_1 = f_2 = 1$ и~$f_{n + 1} = f_n + f_{n - 1}$.
Докажите, что последовательность последних цифр этих чисел периодична, причем
не~имеет предпериода.

\item
Докажите, что цифры после запятой у~числа $a / n$ образуют периодическую
последовательность.

\item
Докажите, что последовательность остатков при делении $16^n + 26^n$ на~$2015$
периодична.

\item
Докажите, что последовательность последних цифр $n^n$ периодична.

\item
Станок выпускает детали двух типов.
На~ленте его конвейера выложены в~одну линию $75$~деталей.
Пока конвейер движется, на~станке готовится деталь того типа, которого на~ленте
меньше.
Каждую минуту очередная деталь падает с~ленты, а~подготовленная кладется
в~ее~конец.
Через некоторое число минут после включения конвейера может случиться так, что
расположение деталей на~ленте впервые повторит начальное.
Найдите:
\\
\subproblem наименьшее такое число;
\quad
\subproblem все такие числа.

\item
Докажите, что для любого $m$ найдется число Фибоначчи, делящееся на~$m$.

\item
На~бесконечной в~обе стороны ленте записан текст на~русском языке.
Известно, что в~этом тексте число различных кусков из $15$~символов равно числу
различных кусков из $16$~символов.
Докажите, что на~ленте записан периодический в~обе стороны текст, например:
<<...мама мыла раму мама мыла раму...>>.

\item
Назовем сочетанием цифр несколько цифр, записанных подряд.
В~стране Роботландии некоторые сочетания цифр объявлены запрещенными.
Известно, что запрещенных сочетаний конечное число и~существует бесконечная
десятичная дробь, не~содержащая запрещенных сочетаний.
Докажите, что существует бесконечная периодическая десятичная дробь,
не~содержащая запрещенных сочетаний.

\item
По~кругу расставлено несколько коробочек.
В~каждой из~них может лежать один или несколько шариков (или она может быть
пустой).
Ход состоит в~том, что из~какой-то коробочки берутся все шарики
и~раскладываются по~одному, двигаясь по~часовой стрелке, начиная со~следующей
коробочки.
\\
\subproblem
Пусть на~каждом следующем ходу разрешается брать шарики из~той коробочки,
в~которую был положен последний шарик на~предыдущем ходу.
Докажите, что в~какой-то момент повторится начальное расположение шариков.
\\
\subproblem
Пусть теперь на~каждом ходу разрешается брать шарики из~любой коробочки.
Верно~ли, что за~несколько ходов из~любого начального расположения шариков
по~коробочкам можно получить любое другое?

\end{problems}

