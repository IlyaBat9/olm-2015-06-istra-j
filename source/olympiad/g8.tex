% $date: 2015-06-01
% $timetable:
%   g8r2:
%     2015-06-01:
%       2:
%       3:
%   g8r1:
%     2015-06-01:
%       2:
%       3:
% $groups$delegate: {not: {or: [g8r2, g8r1]}}

\section*{Вступительная олимпиада}

% $authors:
% - Глеб Погудин
% - Илья Богданов


\subsection*{Довывод}

\begin{problems}
	
\item
Отец и сын катаются на коньках по кругу.
Время от времени отец обгоняет сына.
После того, как сын переменил направление своего движения на противоположное,
они стали встречаться в $5$ раз чаще.
Во сколько раз отец быстрее сына?
% можно заменить на любую несложную приятную на вид задачу

\item
В гости к Андрею пришло $10$ человек, у всех разный размер ноги.
Они уходят по одному, причем каждый надевает произвольную пару обуви, в которую
может влезть.
В какой-то момент оказалось, что больше никто уйти не может, так как имеющаяся
обувь слишком мала для всех оставшихся.
Какое наибольшее число человек могло остаться?

\item
У двух равнобедренных треугольников равны основания.
Более того, оказалось, что их вершины лежат все на одной окружности.
Обязательно ли эти треугольники равны?

\item
Докажите, что $2^{27} + 3^{27}$ делится на $5$.
% тут интересно посмотреть, как они будут решать --- через последнюю цифру,
% через формулы сокращенного умножения или еще как.
% можно и усложнить, заменив на 2^{54} + 3^{54} делится на 13.

\item
<<Крокодилом>> называется фигура, ход которой заключается в прыжке на клетку,
в которую можно попасть сдвигом на одну клетку по вертикали или горизонтали,
а затем на $N$ клеток в перпендикулярном направлении
(при $N = 2$ <<крокодил>>~--- это шахматный конь).
При каких $N$ <<крокодил>> может пройти с каждой клетки бесконечной шахматной
доски на любую другую?
% можно сделать downgrade и попросить обойти для N = 8 — будет просто
% конструктив

\end{problems}


\subsection*{Вывод}

\begin{problems}

\item
Какое наибольшее количество чисел от $1$ до $2015$ можно выбрать так, чтобы
у любых двух выбранных чисел наибольший общий делитель был больше $1$?

\item
\label{olympiad/g8:problem:bottles}%
На дно горизонтальной прямоугольной коробки кладут бутылки
(тоже горизонтально).
В коробке достаточно места, чтобы в нижнем ряду поместилось три бутылки,
но не хватает места для четырех.
Поверх этих трех бутылок во второй ряд естественным образом кладут две бутылки.
В третий ряд кладут три бутылки.
Докажите, что центры бутылок третьего слоя лежат на одной прямой
(рис.~\ref{olympiad/g8:problem:bottles:fig}).

\begin{figure}[ht]\begin{center}
    \jeolmfigure[width=0.5\linewidth]{bottles}
    \caption{к задаче~\ref{olympiad/g8:problem:bottles}}
    \label{olympiad/g8:problem:bottles:fig}
\end{center}\end{figure}

\item
На столе стоят $n$ ящиков, занумерованных числами от $1$ до $n$.
По ним произвольным образом разложено $n$ шариков.
За ход разрешается сделать следующее: если в одном из крайних ящиков есть
шарик, то можно переложить его в соседний ящик;
если в одном из остальных ящиков есть хотя бы два шарика, то один можно
переложить в соседний слева ящик, а другой --- в соседний справа.
Докажите, что вне зависимости от исходного расположения шариков, можно сделать
так, чтобы в каждом ящике лежал ровно один шарик.
% не сложновато ли?

\end{problems}

