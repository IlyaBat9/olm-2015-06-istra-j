% $date: 2015-06-13
% $timetable:
%   g8r1:
%     2015-06-13:
%       1:
%       2:

\section*{Зачёт}

% $authors:
% - Андрей Меньщиков
% - Глеб Погудин

\begin{problems}

\item
Найдите все пары простых чисел $p > q$ такие, что $p + 3 q + 1$ делится
на~$(p - q)$.

\item
Каждый зритель, купивший билет в~первый ряд кинотеатра, занял одно из~мест
в~первом ряду.
Оказалось, что все места в~первом ряду заняты, но~каждый зритель сидит
не~на~своем месте.
Билетёр может менять местами соседей, если оба сидят не~на~своих местах.
Всегда~ли он~может рассадить всех на~свои места?

\item
Дан треугольник $ABC$.
Внутри него взяли точку~$M$ и~соединили ее~с~вершинами.
Получилось три треугольника.
Найдите геометрическое место точек~$M$, для которых сумма площадей двух из~этих
треугольников будет равна площади третьего.

\item
Десять футбольных команд сыграли каждая с~каждой по~одному разу.
В~результате у~каждой команды оказалось ровно по $N$~очков.
Каково наибольшее возможное значение~$N$?
(Победа~--- 3~очка, ничья~--- 1~очко, поражение~--- 0).

\item
На~сторонах выпуклого четырехугольника как на~диаметрах построены четыре круга.
Докажите, что они покрывают весь четырехугольник.

\item
Число $x + \frac{1}{x}$~--- целое.
Докажите, что для любого натурального~$n$ число $x^n + \frac{1}{x^n}$ также
является целым.

\item
Сколькими способами можно расставить нули и~единицы в~таблице
$2015 \times 2015$ так, чтобы в~каждой строке и~в~каждом столбце было нечетное
число единиц?

\item
Точка~$K$~--- середина гипотенузы~$AB$ прямоугольного треугольника $ABC$.
На~катетах $AC$ и~$BC$ выбраны точки $M$ и~$N$ соответственно так, что
$\angle MKN = 90^{\circ}$.
Докажите, что из~отрезков $AM$, $BN$ и~$MN$ можно составить прямоугольный
треугольник.

\item
Натуральные числа $a$, $b$, $c$ таковы, что $a^{804} + b^{804} + c^{804}$
делится на~2015.
Докажите, что число $a b c$ также делится на~2015.

\end{problems}

