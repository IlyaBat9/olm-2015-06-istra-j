% $date: 2015-06-05
% $timetable:
%   g8r1:
%     2015-06-05:
%       1:

\section*{НОД и НОК. Алгоритм Евклида}

% $authors:
% - Андрей Меньщиков

%\item
%Докажите, что $(a, b) \cdot [a, b] = a \cdot b$.

%\item
%Докажите, что числа $2 n + 13$ и $n + 7$ взаимно просты при любом
%натуральном~$n$.

\claim{Алгоритм Евклида}
Нахождение НОДа двух натуральных чисел $a$ и $b$.
\par
Если $a = b$, то $(a, b) = a$.
Иначе будем считать, что $a > b$.
Обозначим через $a_0$ число $a$, а через $a_1$ число $b$.
Делим с остатком:
\begin{gather*}
    a_0 = q_1 a_1 + a_2
,\quad
    a_1 = q_2 a_2 + a_3
,\quad\ldots,\quad
    a_{n-1} = q_n a_n
.\end{gather*}
Тогда $(a, b) = a_n$.

\claim{Упражнение 1}\setcounter{jeolmsubproblem}{0}
Докажите, что
\quad
\subproblem $(a, b) = (a - b, b)$;
\quad
\subproblem $(a c, b c) = c \cdot (a, b)$.

\claim{Упражнение 2: линейное разложение НОД}
Докажите, что для любых целых $a$ и $b$ найдутся целые $x$ и $y$ такие, что
$a x + b y = (a, b)$.

\begin{problems}

\item
Петя посчитал НОК всех чисел от 1 до 1000, а Вася~--- всех чисел
от 501 до 1000.
У кого результат получился больше и во сколько раз?

\item
На складе лежат в большом количестве ширлы, мырлы и дырлы.
Ширла состоит из пяти шашек, мырла~--- из семи машек,
дырла~--- из девяти дашек.
Все шашки одинаковы, машки -- тоже, одинаковы и все дашки.
У Васи есть чашечные весы без гирь, и он хочет за одно взвешивание узнать, что
тяжелее: две шашки или машка с дашкой.
Все изделия, имеющиеся на складе, не разбираются.
Что же делать Васе?

\item
Даны 6 натуральных чисел.
Для каждой пары их НОД записали на доске.
Может ли оказаться так, что на доске записаны все числа от 1 до 15?

\iffalse
\item
Про натуральные числа $a$ и $b$ известно, что $a^2 + b^2$ делится на $a b$.
Докажите, что $a = b$.
\fi

\item
Найдите
\quad
\subproblem $(99! + 100!, 101!)$;
\quad
\subproblem $(\underbrace{11\ldots 1}_{100}, \underbrace{11\ldots 1}_{2015})$.

\item
Какие значения может принимать $(3 n + 2, 10 n + 23)$?

\item
Докажите, что $(a^m - 1, a^n - 1) = a^{(m,n)} - 1$.

\item
По бесконечной шахматной доске ходит $(m, n)$-крокодил, который может за один
ход сдвинуться на $m$~клеток по горизонтали или вертикали, а затем~---
на $n$~клеток в перпендикулярном направлении.
При каких $m$ и $n$ такой $(m,n)$--крокодил сможет попасть из любой клетки
доски в любую другую?

\item
Числа $m$ и $n$ взаимно просты.
Какое наибольшее значение может принимать $(m + 2015 n, n + 2015 m)$?

\item
Докажите, что
\\
\subproblem
$(f_m, f_n) = 1$, где $f_k = 2^{2^k} + 1$~--- числа Ферма;
\\
\subproblem
число $(2^{2^n} - 1)$ имеет по крайней мере $n$ различных простых делителей.

\end{problems}

\claim{Замечание}
Из предыдущей задачи следует, что простых чисел бесконечно много.

\begin{problems}

\item
\subproblem
Докажите, что если для некоторых натуральных $a$ и $b$ верно, что
\[
    \text{НОК}(a, a + 5) = \text{НОК}(b, b + 5)
\text{,\quad то\enspace}
    a = b
\,.\]
\subproblem
Может ли при различных натуральных $a$, $b$ и $c$ выполняться равенство
\[
    \text{НОК}(a, b) = \text{НОК}(a + c, b + c)
\,?\]

\end{problems}

