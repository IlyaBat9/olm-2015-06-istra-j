% $date: 2015-06-06
% $timetable:
%   g8r1:
%     2015-06-06:
%       1:

\section*{Линейные диофантовы уравнения}

% $authors:
% - Андрей Меньщиков

\theorem
Пусть $a$ и~$b$~--- взаимно простые числа.
Тогда уравнение $a x + b y = c$ имеет бесконечно много целочисленных решений,
причем все они имеют вид
\[
    \left\{\begin{aligned} &
        x = x_0 + b k
    , \\ &
        y = y_0 - a k
    , \end{aligned}\right.
\]
где $k$~--- произвольное целое число, а~$(x_0, y_0)$~--- одно из~решений
уравнения $a x + b y = c$. 

Для того, чтобы решить в~целых числах линейное уравнение
$a x + b y = c$ ($a, b \neq 0$), надо действовать по~следующему алгоритму:
\par
\textbf{(1)} Находим $d = \text{НОД}(a, b)$.
\par
\textbf{(2)}
Проверяем, делится~ли $c$ на~$d$.
Если нет, то~уравнение не~имеет решений.
Если да, то~делим обе части на~$d$, переходя к~равносильному уравнению
$a' x + b' y = c'$ со~взаимно простыми $a' = a / d$ и~$b' = b / d$
и~правой частью $c' = c / d$.
\par
\textbf{(3)}
Находим частное решение $(x_0, y_0)$ уравнения $a' x + b' y = c'$
(если получается~--- подбором, если нет~--- с~помощью алгоритма Евклида).
\par
\textbf{(4)}
Записываем ответ в~виде
\[
    \left\{\begin{aligned} &
        x = x_0 + b k
    , \\ &
        y = y_0 - a k
    , \end{aligned}\right.
\quad
    k \in \mathbb{Z}
\,.\]

\begin{problems}

\item
Решите в~целых числах уравнение
\\
\sp $9 x - 5 y = 6$;
\quad
\sp $2007 x + 2015 y = 257$.

\item
Петя задумал натуральное число, умножил его на~51, затем поделил с~остатком
на~714 и~получил в~остатке 13.
Не~ошибся~ли он?

\item
Найдите две разные обыкновенные дроби~--- одну со~знаменателем 8, другую
со~знаменателем 13, чтобы модуль разности между ними был как можно меньше.

\item
Натуральные числа $a$ и~$b$ взаимно просты.
Докажите, что уравнение $a x + b y = a b$ не~имеет решений в~натуральных
числах.

\item
В~клетчатом прямоугольнике $m \times n$ провели диагональ.
Сколько клеток она пересекла?

\item
Ширлы и~мырлы стоят целое число копеек.
Известно, что 175 ширл стоят дороже, чем 125 мырл, но~дешевле, чем 126 мырл.
Хватит~ли Васе одного рубля на~трех ширл и~одну мырлу?

\item
Натуральные $a$, $b$ и~$c$ таковы, что $a b + b c = a c$.
Докажите, что
\[
    \text{НОК}(a, b) = \text{НОК}(b, c) = \text{НОК}(a, c)
\,.\]

\item
Верно~ли, что для любых натуральных $a$ и~$b$ найдутся натуральные $p$ и~$q$
такие, что при любом натуральном $n$ дробь $\frac{a n + p}{b n + q}$
несократима?

\item
Найдите все целочисленные решения уравнений
\\
\sp
$2 x + 3 y + 5 z = 11$;
\quad
\sp
$12 x + 15 y + 20 z = 4$.

\item
Сформулируйте алгоритм решения в~целых числах уравнений вида
\[
    a x + b y + c z = d
\,.\]

\item
Последовательность $x_1, x_2, \ldots$ задана правилами:
$x_1 = 2$, $x_{n+1}$~--- наибольший простой делитель числа
$x_1 \cdot x_2 \cdot \ldots \cdot x_n + 1$ при всех $n \geq 1$.
Докажите, что $x_n \neq 5$ ни~при каком натуральном $n$. 

%\item
%Про натуральные числа $a$ и~$b$ известно, что $a^2 + b^2$ делится на~$a b$.
%Докажите, что $a = b$.

\end{problems}

