% $date: 2015-06-07
% $timetable:
%   g8r1:
%     2015-06-07:
%       2:

\section*{Малая теорема Ферма и теорема Эйлера}

% $authors:
% - Андрей Меньщиков

\begingroup \def\eulerphi{\mathrm{\phi}}

\definition
Набор чисел называется \emph{полной системой вычетов} по~модулю $m$, если в~нем
встречаются все возможные остатки от~деления на~$m$ ровно по~одному разу.

\definition
Набор чисел называется \emph{приведённой системой вычетов} по~модулю $m$, если
в~нем встречаются все возможные взаимно простые c $m$ остатки от~деления ровно
по~одному разу.

\mbox{\claim{Малая теорема Ферма (1)}}%
Пусть $p$~--- простое число, а~$a$ не~делится на~$p$.
Тогда $(a^{p-1} - 1)$ делится на~$p$.

\mbox{\claim{Малая теорема Ферма (2)}}%
Пусть $a$~--- целое число, $p$~--- простое.
Тогда $(a^p - a)$ делится на~$p$.

\begin{problems}

\item
\subproblem
Вспомните бином Ньютона
\[
    (a + b)^n
=
    a^n + n a^{n-1} b +
    \ldots +
    C_{n}^{k} a^{n-k} b^k +
    \ldots +
    n a b^{n-1} + b^n
\, , \] а~затем докажите, что $C_{p}^{l}$ делится на~$p$, где $p$~--- простое
число, а~$l$~--- натуральное, меньшее $p$.
\\
\subproblem
Докажите, что $\bigl( (a + 1)^p - a^p - 1 \bigr)$ делится на~$p$ и~докажите
малую теорему Ферма по~индукции.

\item
\label{algebra/number-theory/fermats-theorem-g8r1:problem:fermat-proof-1}%
\claim{Воспоминание}
Пусть $a$ не~делится на~$p$.
Тогда множество остатков чисел
\[
    1 \cdot a, 2 \cdot a, \ldots, (p - 1) \cdot a
\]
при делении на~$p$ совпадает с~множеством $\{ 1, 2, \ldots, p - 1 \}$.
\par
С~помощью этого воспоминания докажите малую теорему Ферма.

\item
\label{algebra/number-theory/fermats-theorem-g8r1:problem:fermat-proof-2}%
Каждой вершине правильного $(p - 1)$-угольника сопоставим какой-то ненулевой
остаток при делении на~$p$.
Проведем из~остатка~$k$ стрелку в~остаток~$k a$.
\\
\subproblem
Докажите, что из~каждой точки выходит одна стрелка, и~в~каждую точку входит
одна стрелка, откуда следует, что все стрелки разбиваются на~циклические
маршруты.
\\
\subproblem
Докажите, что у~всех этих циклических маршрутов одна и~та~же длина, делящая
$(p - 1)$, а~затем выведите отсюда малую теорему Ферма.

\item
Сколько существует способов раскрасить вершины правильного $p$-угольника
в~$a$ цветов?
(Раскраски, которые можно совместить поворотом, считаются одинаковыми.)
Найдите формулу и~выведите из~нее малую теорему Ферма.

\end{problems}

\definition
Пусть $n$~--- натуральное число.
\emph{Функция Эйлера} $\eulerphi(n)$ определяется как количество натуральных
чисел, не~превосходящих $n$, взаимно простых с~$n$.

\claim{Упражнение}
Докажите, что если $n > 2$, то~$\eulerphi(n)$ четно.

\theoremof{Эйлера}
$a$ и~$n$~--- натуральные взаимно простые числа.
Тогда $(a^{\eulerphi(n)} - 1)$ делится на~$n$.

\begin{problems}

\item
\subproblem
Приведите рассуждения, аналогичные
задаче~\ref{algebra/number-theory/fermats-theorem-g8r1:problem:fermat-proof-1},
и~докажите теорему Эйлера.
\\
\subproblem
Приведите рассуждения, аналогичные
задаче~\ref{algebra/number-theory/fermats-theorem-g8r1:problem:fermat-proof-2},
и~докажите теорему Эйлера.

\item
\subproblem
Пусть $p$ и~$q$~--- различные простые числа, $\alpha$~--- натуральное.
Найдите $\eulerphi(p^\alpha)$ и~$\eulerphi(p q)$.
\\
\subproblem
Докажите, что если $a$ и~$b$ взаимно просты,
то~$\eulerphi(a \cdot b) = \eulerphi(a) \cdot \eulerphi(b)$.
\\
\subproblem
Пусть
\(
    n
=
    p_1^{\alpha_1} \cdot p_2^{\alpha_2} \cdot \ldots \cdot p_k^{\alpha_k}
\)~---
разложение $n$ на~простые множители.
Докажите, что
\[
    \eulerphi(n)
=
    n
    \left(
        1 - \frac{1}{p_1}
    \right)
    \left(
        1 - \frac{1}{p_2}
    \right)
    \ldots
    \left(
        1 - \frac{1}{p_k}
    \right)
\; . \]

\item
\claim{Усиление теоремы Эйлера}
Пусть $m = p_1^{\alpha_1} p_2^{\alpha_2} \ldots p_k^{\alpha_k}$~--- разложение
$m$ на~простые множители,
\[
    x
=
    \operatorname{\text{НОК}} \bigl(
        \eulerphi(p_1^{\alpha_1}),
        \eulerphi(p_2^{\alpha_2}),
        \ldots,
        \eulerphi(p_k^{\alpha_k})
    \bigr)
\, . \]
Докажите, что для любого $a$, взаимно простого с~$m$, выполняется сравнение
$a^x \equiv 1 \pmod m$.

\end{problems}

\endgroup % \def\eulerphi

