% $date: 2015-06-04
% $timetable:
%   g9r2:
%     2015-06-04:
%       1:

\section*{Показатели}

% $authors:
% - Владимир Брагин

\definition
Зафиксируем взаимно простые числа $a$ и~$n$.
Пусть $d$~--- наименьшее такое натуральное число, что $(a^d - 1)$ делится
на~$n$.
Число~$d$ называется \emph{показателем~$a$ по~модулю~$n$}.

\begin{problems}

\item
\sp
Пусть $d$~--- показатель~$a$ по~модулю~$n$.
Пусть $a^l \equiv 1 \pmod{n}$.
Докажите, что $d \mid l$.
\\
\sp
Докажите, что $a^s \equiv a^r \pmod{n}$ тогда и~только тогда, когда
$s \equiv r \pmod{d}$.

\item
Докажите, что показатель любого числа по~модулю~$n$ (взаимно простого с~$n$,
конечно) делит $\phi(n)$.

\item
Докажите, что если $a > 1$, то~$n$ делит $\phi(a^n - 1)$.

\item
\sp
Пусть $p$~--- простое число.
Докажите, что любой простой делитель числа $(a^p - 1)$ или делит $(a - 1)$ или
имеет вид $2 p x + 1$.
\\
\sp
Выведите отсюда, что простых чисел вида $2 p k + 1$ бесконечно много.

\item
\sp
Докажите, что любой нечетный простой делитель числа $a^{2^{k}} + 1$ имеет вид
$2^{k+1} x + 1$.
\\
\sp
Докажите, что простых чисел вида $2^{k+1} x + 1$ бесконечно много.

\item
Пусть $p$ и~$q$~--- простые числа, большие 5.
Докажите, что если $p \mid 2^q + 3^q$, то~$q < p$.

\item
Докажите, что если $n > 1$, то~$(2^n - 1)$ не~делится на~$n$. 

\item
Сколько различных остатков принимают степени двойки при делении на~$3^k$?

\end{problems}

