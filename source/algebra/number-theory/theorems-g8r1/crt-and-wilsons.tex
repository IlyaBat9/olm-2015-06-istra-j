% $date: 2015-06-09
% $timetable:
%   g8r1:
%     2015-06-09:
%       2:

\section*{КТО и Вильсон}

% $authors:
% - Андрей Меньщиков


\claim{Упражнение 1}
Найдите наименьшее натуральное число, дающее при делении на~$2$, $3$, $5$, $7$
остатки $1$, $2$, $4$, $6$ соответственно.

\claim{Упражнение 2}\setcounter{jeolmsubproblem}{0}
\sp
Решите в~целых числах уравнение $10 x + 8 = 11 y + 10$.
\\
\sp
Найдите все числа, дающие при делении на~8 остаток~2, а~при делении на~13~---
остаток~11.

\claim{Китайская теорема об остатках}
Пусть целые числа $m_1, \ldots, m_n$ попарно взаимно просты,
$m = m_1 \cdot \ldots \cdot m_n$, и~$a_1, \ldots, a_n$~--- произвольные целые
числа.
Тогда существует единственное целое число $x$ такое, что $0 \leq x < m$ и
\[
\left\{\begin{aligned} &
    x \equiv a_1 \pmod{m_1}
, \\ &
    x \equiv a_2 \pmod{m_2}
, \\ & \ldots \\ &
    x \equiv a_n \pmod{m_n}
. \end{aligned}\right.
\]

\begin{problems}

\item
Пусть натуральные числа $m_1, m_2, \ldots, m_n$ попарно взаимно просты.
Докажите, что если числа $x_1, x_2, \ldots, x_n$ пробегают полные системы
вычетов по~модулям $m_1, m_2, \ldots, m_n$ соответственно, то~число
\[
    x
=
    x_1 m_2 \ldots m_n +
    m_1 x_2 \ldots m_n +
    \ldots +
    m_1 m_2 \ldots m_{n-1} x_n
\]
пробегает полную систему вычетов по~модулю
$m_1 \cdot m_2 \cdot \ldots \cdot m_n$.
Выведите отсюда КТО.

\item
Проникнувшись упражнением 2, докажите КТО по~индукции.

\item
\sp
Натуральные числа $m_1, \ldots, m_n$ попарно взаимно просты.
Докажите, что число
\(
    x = (m_2 \cdot m_3 \cdot \ldots \cdot m_n)^{\phi(m_1)}
\)
является решением системы
\[
\left\{\begin{aligned} &
    x \equiv 1 \pmod{m_1}
, \\ &
    x \equiv 0 \pmod{m_2}
, \\ & \ldots \\ &
    x \equiv 0 \pmod{m_n}
. \end{aligned}\right.
\]
\sp
Найдите в~явном виде число~$x$, удовлетворяющее КТО.

\end{problems}

\theoremof{Вильсона}
Пусть $p$~--- простое число.
Тогда $(p - 1)! \equiv -1 \pmod{p}$.

\claim{Упражнение}
Натуральное $p > 1$ таково, что $(p - 1)! \equiv -1 \pmod{p}$.
Докажите, что $p$~--- простое число.

\begin{problems}

\item
Сколько существует замкнутых ломаных, проходящих через все вершины правильного
$p$-угольника?
(Ломаные, которые можно совместить поворотом, считаются одинаковыми.)
Найдите формулу и~выведите из~неё теорему Вильсона.

\item
Пусть целое число~$a$ не~делится на~простое~$p$.
Докажите, что найдётся целое число~$b$ такое, что $a b \equiv 1 \pmod{p}$,
а~затем выведите отсюда теорему Вильсона.

\item
Генерал построил солдат в~колонну по~4, но~при этом солдат Иванов остался
лишним.
Тогда генерал построил солдат в~колонну по~5.
И~снова Иванов остался лишним.
Когда~же и~в~колонне по~6 Иванов оказался лишним, генерал посулил ему наряд вне
очереди, после чего в~колонне по~7 Иванов нашёл себе место и~никого лишнего
не~осталось.
Сколько солдат могло быть у~генерала?

\item
Докажите, что для любого натурального~$k$ существует сколь угодно длинный
отрезок из~натуральных чисел, каждое из~которых делится по~крайней мере на~$k$
различных простых чисел.

\item
Пусть $p$~--- простое число.
Докажите, что $(2 p - 1)! - p$ делится на~$p^2$.

\end{problems}

