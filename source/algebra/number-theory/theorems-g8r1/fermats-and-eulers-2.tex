% $date: 2015-06-08
% $timetable:
%   g8r1:
%     2015-06-08:
%       1:

\section*{Ферма и Эйлер. Продолжение}

% $authors:
% - Андрей Меньщиков

% $build$matter[print]: [[.], [.]]

\begin{problems}

\item
Найдите остаток при делении $9^{2015}$ на
\quad
\sp 23
\quad
\sp 1001.

\item
Пусть $p$ и~$q$ -- различные простые числа.
Докажите, что
\(
    p^q + q^p
\equiv
    p + q
\pmod{p q}
\).

%\item
%Докажите, что $2^{3^k} + 1$ делится на~$3^{k+1}$.

\item
Докажите, что для нечётного натурального $n$ число $(2^{n!} - 1)$ делится
на~$n$.

\item
Докажите, что число $2015^{2016} + 2016^{2015}$ -- составное.

\item
Докажите, что для составного числа $561$ справедлив аналог малой теоремы Ферма:
если $(a, 561) = 1$, то~$a^{560} \equiv 1 \pmod{561}$.

\item
Докажите, что если $p$~--- простое число, отличное от~2 и~5, то~длина периода
разложения $1 / p$ в~десятичную дробь делит $(p - 1)$.

\item
\sp
Докажите, что $(n^{84} - n^4)$ делится на~6800 для любого натурального~$n$.
\\
\sp
Можно~ли 6800 заменить на~какое-то большее число, чтобы утверждение осталось
верным?

\item
Дана последовательность $a_n = 1^n + 2^n + 3^n + 4^n + 5^n$.
Существуют~ли 5 идущих подряд её~членов, делящихся на~2015?

\item
Пусть $p$ и~$p + 2$ -- простые числа.
Докажите, что $2 p (p + 1) (p + 2)$ является общим делителем чисел
$(p^{p+2} - p)$ и~$\bigl( (p + 2)^p - p - 2 \bigr)$.

\item
Пусть $p > 5$ -- простое число.
\\
\sp
Докажите, что число $\underbrace{111 \ldots 11}_{p - 1}$ делится на~$p$.
\\
\sp
Докажите, что число
\[
    \underbrace{1 \ldots 1}_{p}
    \underbrace{2 \ldots 2}_{p}
    \underbrace{3 \ldots 3}_{p}
    \ldots
    \underbrace{9 \ldots 9}_{p}
    - 123456789
\]
делится на~$p.$

%\item
%Докажите, что для любого натурального~$n$ существует число с~суммой цифр $n$,
%делящееся на~$n$.

\end{problems}

