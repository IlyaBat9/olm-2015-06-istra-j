% $date: 2015-06-04
% $timetable:
%   g8r1:
%     2015-06-04:
%       2:

\section*{Сравнения}

% $authors:
% - Андрей Меньщиков

\begin{problems}

\item
Натуральные числа $a$ и~$b$ таковы, что $3 a + 7 b$ дает остаток 5 при делении
на~11.
А~какой остаток дает $2 a + b$ при делении на~11?

\item
Число $N = (16 a + 17 b) \cdot(17 a + 16 b)$ делится на~11.
Докажите, что $N$ делится на~121.

\item
У~числа $2015^{2015}$ нашли сумму цифр.
У~результата опять нашли сумму цифр, и~т.~д., пока не~получилась цифра.
А~что это за~цифра?

\item
У~числа $n^2 + 2 n$ последняя цифра равна 4.
А~какая у~него предпоследняя цифра?

\item
Докажите, что разность произведения всех нечетных чисел от~1 до~2015
и~произведения всех четных чисел от~2 до~2016 делится на~2017.

\item
Пусть $p$~--- простое число, $k$~--- натуральное, не~делящееся на~$p$.
Докажите, что среди остатков чисел
$0 \cdot k$, $1 \cdot k$, $2 \cdot k$, \ldots, $(p - 1) \cdot k$ при делении
на~$p$ все возможные остатки встречаются ровно по~одному разу. 

\item
Пусть $p$ и~$q$~--- последовательные нечетные числа.
Докажите, что $p^p + q^q$ делится на~$p + q$. 

\item
Для натуральных $a$ и~$n$ докажите, что
$a^{2n+3} + (a - 1)^n$ делится на~$(a^2 - a + 1)$.

%\item
%При каких натуральных $n$ число $n^4 + n^3 + n^2 + n + 1$ делится на~2017?

\item
При каких целых $k$ число $a^3 + b^3 + c^3 - k a b c$ делится на~$a + b + c$
при любых целых $a$, $b$, $c$, сумма которых не~равна 0?

%\item
%Теорема Вильсона?

\item
Последовательность задана следующим образом: $a_1 = a_2 = 1$,
$a_{k+2} = a_k \cdot a_{k+1} + 1$ при всех натуральных $k$.
Докажите, что при любом натуральном $n \geq 7$ число $(a_n - 3)$ является
составным.

\end{problems}

