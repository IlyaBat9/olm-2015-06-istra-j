% $date: 2015-06-11
% $timetable:
%   g8r1:
%     2015-06-11:
%       1:

\section*{Числовой разнобой}

% $authors:
% - Андрей Меньщиков

\begin{problems}

\item
Упростите выражение 
\\
\subproblem
\(
    2014 \cdot (2015^9 + 2015^8 + \ldots + 2015^2 + 2016)
\);
\\[0.5ex]
\subproblem
\(\displaystyle
    \frac{1}{1 \cdot 2} +
    \frac{1}{2 \cdot 3} +
    \ldots +
    \frac{1}{2015 \cdot 2016}
\)\,;
\qquad
\subproblem
\(
    1 \cdot 1! +
    2 \cdot 2! +
    \ldots +
    2015 \cdot 2015!
\).

\item
Являются~ли следующие числа точными квадратами?
\\
\subproblem
$3^{40} + 6^{20} + 2^{38}$;
\\
\subproblem
$2013 \cdot 2014 \cdot 2015 \cdot 2016 + 1$;
\\
\subproblem
$2015^2 + 2015^2 \cdot 2016^2 + 2016^2$;
\\
\subproblem
$2015^{2016} + 2015^{1008} + 1$.

\item
Можно~ли из~произведения $1! \cdot 2! \cdot \ldots \cdot 100!$ вычеркнуть один
сомножитель так, что произведение оставшихся будет точным квадратом?

\item
Известно, что числа $p$ и~$p^2 + 2$~--- простые.
Докажите, что число $p^3 + 2$ также является простым.

\item
Докажите, что произведение $n$ последовательных натуральных чисел делится
на~$n!$.

%\item
%Докажите, что простых чисел вида $4 k + 3$ бесконечно много.

\item
Докажите, что уравнение $x^2 + y^2 = 7 (z^2 + t^2)$ не~имеет решений
в~натуральных числах.

\item
Докажите, что для любого натурального $n > 1$ число $n^4 + 4$ является
составным.

\item
Решите в~натуральных числах уравнение $x^4 - 2 y^2 = 1$.

%\item
%\emph{Собственным делителем} натурального числа называется любой его делитель,
%отличный от~1 и~самого этого числа.
%Найдите все натуральные числа, у~которых самый большой собственный делитель
%на~1 больше, чем квадрат самого маленького собственного делителя.

\item
Натуральные числа $a$, $b$, $c$ таковы, что $a^{128} + b^{128} + c^{128}$
делится на~257.
Докажите, что число $a b c$ делится на~257.

\item
Число $(1 + \sqrt{2})^{2015}$ записали в~виде $a + b \sqrt{2}$ с~целыми
$a$ и~$b$.
Докажите, что $|a^2 - 2 b^2| = 1$.

\end{problems}

