% $date: 2015-06-05
% $timetable:
%   g8r2:
%     2015-06-05:
%       2:

\section*{НОД и НОК}

% $authors:
% - Ольга Телешева

%\begin{problems}

%\item
%У~Оли и~Андрея есть 30 булочек и~84 конфеты.
%Они хотят раздать их~как можно большему количеству детей, причем чтобы того
%и~другого было поровну.
%Скольким детям они могут раздать вкусности?

%\end{problems}

\definition
\emph{Наибольшим общим делителем} нескольких натуральных чисел называется
наибольшее число, на~которое делится каждое из~этих чисел.
Обозначение $\text{НОД}(a, b) = (a, b)$.

\definition
Натуральные числа называются \emph{взаимно простыми}, если у~них нет общих
делителей, кроме~1.
То есть числа $a$ и~$b$ взаимно просты, если $(a, b) = 1$.

\definition
\emph{Наименьшим общим кратным} нескольких натуральных чисел называется
наименьшее число, которое делится на~каждое из~этих чисел.
Обозначение $\text{НОК}(a, b) = [a, b]$.

\begin{problems}

%\item
%Найдите:
%\\
%\subproblem $\text{НОД}(48, 270)$;
%\\
%\subproblem $\text{НОД}(525, 990)$;
%\\
%\subproblem $\text{НОД}(99! + 100!, 101!)$.

\item
Докажите, что $\text{НОК}(a, b) \cdot \text{НОД}(a, b) = a b$.

\item
Петя посчитал НОК всех чисел от~1 до~1000, а~Вася НОК всех чисел от~501
от~1000.
У~кого результат получился больше и~во~сколько раз?

\item
Найдите наибольшее трехзначное число~$n$, для которого
$\text{НОД}(n, 1080) = 36$.
% Ответ: 828 = 36 ⋅ 23

\item
Даны шесть натуральных чисел.
Для каждой пары чисел посчитали их~НОД.
Могут~ли среди этих НОДов встречаться все натуральные числа от~1 до~15?
% Нет, так как 7 НОДов этих чисел кратны 2, противоречие.

\item
У~Саши есть клетчатый лист бумаги размера $20 \times 75$.
Каждую минуту Саша отрезает от~этого листа квадрат наибольшего возможного
размера и~кладет на~стол.
Какая сторона будет у~самого маленького квадрата на~столе?

\item
По~окружности радиуса~40 катится колесо радиуса~18.
В~колесо вбит гвоздь, который ударяясь об~окружность, оставляет на~ней отметки.
Сколько всего таких отметок оставит гвоздь на~окружности?
Сколько раз прокатится колесо по~всей окружности, прежде чем гвоздь попадет
в~уже отмеченную ранее точку?

\item
В~прямоугольнике с~целыми сторонами $m$ и~$n$, нарисованном на~клетчатой
бумаге, проведена диагональ.
Через какое число узлов она проходит?
На~сколько частей эта диагональ делится линиями сетки?
% ans: (m, n) + 1, m + n - (m, n).

\end{problems}

\subsection*{Алгоритм Евклида}

%\begin{problems}

%\item
%На~прямой сидит блоха, которая может прыгать на 5~см или 7~см вправо или влево.
%Сможет~ли она сместиться после нескольких прыжков вправо на 3~см от~начального
%положения?
%Если сможет, то~как она должна прыгать?

%\item
%Найдите какие-нибудь целые $x$ и~$y$, для которых:
%\\
%\subproblem $12 x - 5 y = 1$;
%\qquad
%\subproblem $12 x + 5 y = 19$.

%\item
%На~доске написаны числа $4800$ и~$3003$.
%Ринат вычисляет разность чисел на~доске и~заменяет любое из~чисел на~эту
%разность.
%\\
%\subproblem
%Может~ли он~с~помощью таких операций на~доске получились числа 18 и~10?
%\\
%\subproblem
%Какое наименьшее положительное число он~может получить?

%\item
%На~доске написаны числа $a$ и~$b$.
%Ринат вычисляет разность чисел на~доске и~заменяет любое из~чисел на~эту
%разность.
%\\
%\subproblem
%Докажите, что все общие делители чисел на~доске всегда одни и~те~же.
%\\
%\subproblem
%Докажите, что Ринат не~сможет получить натуральное число, меньшее
%$\text{НОД}(a, b)$.
%\\
%\subproblem
%Докажите, что после какого-нибудь хода на~доске окажется 0.
%\\
%\subproblem
%Докажите, что вместе с~нулем на~доске присутствует $\text{НОД}(a, b)$.
%% Если есть 0, то~до~этого числа K и K и НОД(a, b) = НОД(K, K) = K.

%\end{problems}

\claim{Упражнение}
\setcounter{jeolmsubproblem}{0}%
Докажите следующие свойства:
%\\
%\subproblem
%Если $m$ делится на~$n$, то~$(n, m) = n;$
%\\
%\subproblem
%$(a \cdot m, a \cdot n) = a \cdot (m, n);$
%\\
%\subproblem
%Если $(a, b) = d$, то~$(\frac{a}{d}, \frac{b}{d}) = 1;$
\\
\subproblem
Если $a \geq b$, то~$\text{НОД}(a, b) = \text{НОД}(b, a - b)$.
\\
\subproblem
Если $a = k b + r$, то~$\text{НОД}(a, b) = \text{НОД}(b, r)$.
\\
\subproblem
Если $b \mid a$, то~$\text{НОД}(a,b) = b$.

\claim{Алгоритм Евклида}
Для того, чтобы найти НОД двух чисел $a$ и~$b$, нужно выполнить последовательно
несколько делений с~остатком:
\begin{gather*}
    a = b q_1 + r_1
, \quad
    b = r_1 q_2 + r_2
, \\
    r_1 = r_2 q_3 + r_3
, \quad
    r_2 = r_3 q_4 + r_4
, \\ \ldots \\
    r_{n-2} = r_{n-1} q_n + r_n
, \quad
    r_{n-1} = r_n q_{n+1} + 0
\,.\end{gather*}
На~каждом шаге предыдущий делитель делится с~остатком на~предыдущий остаток.
Так продолжается до~тех пор, пока на~каком-то шаге остаток не~станет равен 0.
Последний ненулевой остаток равен $\text{НОД}(a, b)$.

\begin{problems}

\item
Найдите, используя алгоритм Евклида:
\\
\subproblem $\text{НОД}(546, 452)$;
\\
\subproblem $\text{НОД}(257, 646)$;
%\\
%\subproblem $\text{НОД}(1026, 998)$;
%\\
%\subproblem $\text{НОД}(478717, 24139)$;
\\
\subproblem
\(
    \text{НОД} (
        \underbrace{11 \ldots 11}_{n},
        \underbrace{11 \ldots 11}_{m}
    )
\).
% 11…11 — ровно (m, n) единичек

%\item
%Найдите какие-нибудь целые $x$ и~$y$, для которых $998 x - 546 y = 2$.

%\item
%От~прямоугольника $2009 \times 35$ по~прямой отрезают квадраты.
%Каждый раз получается прямоугольник, и~от~него можно отрезать квадрат
%со~стороной равной меньшей из~сторон прямоугольника.
%Так делают, пока не~останется квадрат.
%Чему равна сторона оставшегося квадрата?

\item
Найдите $\text{НОД}(2^{80} - 1,  2^{60} - 1)$.

\item
Докажите, что для каждого натурального $n$ дробь $\frac{12 n + 1}{30 n + 2}$
несократима.

%\item
%Найдите пару чисел, не~больших 1000, для которых алгоритм Евклида заканчивает
%работу (получает 0) только через 14 шагов.

%\item
%Какие значения может принимать $\text{НОД}(2 n + 13, n + 7)$, где $n$~---
%целое число.

\item
Найдите $\text{НОД}(11! - 20, 10! - 20)$.

%\item
%Найдите НОД всех шестизначных чисел, составленных из~цифр 1, 2, 3, 4, 5, 6 без
%повторений.

%\item
%Докажите, что $\text{НОД}(a^n - 1, a^m - 1) = a^{\text{НОД}(n, m)} - 1$.

%\claim{Теорема о~линейном представлении НОД}
%Докажите, используя алгоритм Евклида, что если $\text{НОД}(a, b) = d$,
%то~найдутся такие целые $m$ и~$n$, что $d = m a + n b$.

%\item
%Докажите, что если $a$ и~$b$ взаимно просты, то~уравнение $a x + b y = 1$
%разрешимо в~целых числах.

\end{problems}

