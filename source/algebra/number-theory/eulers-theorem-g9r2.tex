% $date: 2015-06-02
% $timetable:
%   g9r2:
%     2015-06-02:
%       1:

% $caption: Классические теоремы теории чисел

\section*{Теория чисел. Вспоминаем (узнаём?) классические теоремы}

% $authors:
% - Владимир Брагин

\claim{Малая теорема Ферма}
Если $p$~--- простое число, а~$a$ не~делится на~$p$, то~$(a^{p-1} - 1)$ делится на~$p$.
\par
\emph{Альтернативная формулировка.}
Если $p$~--- простое число, то~$(a^p - a)$ делится на~$p$.

\begin{problems}

\item
\subproblem
Вспомните, что
\(
    (a + b)^n
=
    a^n + n a^{n-1} b +
    \ldots +
    C_{n}^{k} a^{n-k} b^k +
    \ldots +
    b^n
\).
\\
\subproblem
Осознайте, что $C_{p}^{l}$ делится на~$p$, если $p$~--- простое число,
а~$0 < l < p$.
\\
\subproblem
Докажите, что $\bigl( (a + 1)^p - a^p - 1 \bigr)$ делится на~$p$ и~выведите
отсюда доказательство малой теоремы Ферма по~индукции.

\item
\subproblem
Пусть $a$ не~делится на~$p$.
Докажите, что среди чисел
\[
    1 \cdot a, 2 \cdot a, \ldots, (p - 1) \cdot a
\]
все ненулевые остатки при делении на~$p$ содержатся по~одному разу.
\\
\subproblem
\label{algebra/number-theory/eulers-theorem-g9r2:problem:fermat-proof-1}%
Из~того, что произведение остатков в~одинаковых наборах дают одинаковые
остатки, выведите малую теорему Ферма.

\item
\label{algebra/number-theory/eulers-theorem-g9r2:problem:fermat-proof-2}%
Отметим на~бумаге произвольным образом $(p - 1)$ точку.
Каждой точке сопоставим какой-то ненулевой остаток при делении на~$p$.
Проведем из~остатка~$k$ стрелочку в~остаток~$k a$.
\\
\subproblem
Убедитесь, что из~каждой точки выходит одна стрелочка, и~в~кажду точку входит
одна стрелочка.
\\
\subproblem
Поймите, что тогда все точки разбиваются на~циклические маршруты.
\\
\subproblem
Докажите, что у~всех циклических маршртутов одна и~та~же длина и~она делит
$(p - 1)$.
\\
\subproblem
Выведите отсюда малую теорему Ферма.

\end{problems}

\definition
Пусть $n$~--- натуральное число.
Обозначим $\phi(n)$ количество чисел, не~превосходящих $n$, взаимно простых
с~$n$.
Функция $\phi(n)$ называется \emph{функцией Эйлера}.

\theoremof{Эйлера}
Пусть $n$~--- натуральное число, $a$~--- взаимно простое с~$n$.
Тогда $(a^{\phi(n)} - 1)$ делится на~$n$.

\begin{problems}

\item
Найдите $\phi(p)$, где $p$ простое, $\phi(100)$, $\phi(2^l)$, $\phi(p^k)$.

\item
\subproblem
Докажите, что если умножить все взаимно простые c $n$ остатки на~$a$
(которое с~$n$ тоже взаимно просто), то~получатся все взаимно простые с~$n$
остатки по~одному разу.
\\
\subproblem
Проведите рассуждения, аналогичные
\ref{algebra/number-theory/eulers-theorem-g9r2:problem:fermat-proof-1}
и~докажите теорему Эйлера.
\\
\subproblem
Проведите рассуждения, аналогичные
задаче~\ref{algebra/number-theory/eulers-theorem-g9r2:problem:fermat-proof-2}
и~докажите теорему Эйлера другим способом.

\item
Докажите, что $(n^{561} - n)$ делится на~$561$.

\item
\subproblem
Докажите, что $(n^{84} - n^4)$ делится на~6800 для любого натурального $n$.
\\
\subproblem
Можно~ли вместо 6800 доказать для какого-то большего числа?

\item
Докажите, что $2^{3^k} + 1$ делится на~$3^{k+1}$.

\item
\subproblem
Пусть $p$ и~$q$~--- два различных простых числа.
Докажите, что $\phi(pq) = (p - 1) \cdot (q - 1)$.
\\
\subproblem
Докажите, что если $a$ и~$b$ взаимно просты,
то~$\phi(a \cdot b) = \phi(a) \cdot \phi(b)$.
\\
\subproblem
Пусть
\(
    n
=
    p_1^{\alpha_1} \cdot p_2^{\alpha_2} \cdot \ldots \cdot p_k^{\alpha_k}
\)
(предполагаем, что все $p_i$ различны).
Чему равно $\phi(n) / n$?

\end{problems}

