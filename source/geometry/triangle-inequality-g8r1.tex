% $date: 2015-06-08
% $timetable:
%   g8r1:
%     2015-06-08:
%       2:

\section*{Неравенство треугольника}

% $authors:
% - Илья Богданов

\begin{problems}

\item
Докажите, что длина ломаной, соединяющей концы отрезка, не~меньше длины отрезка.

\item
Докажите, что сумма диагоналей выпуклого четырехугольника больше суммы его двух
противоположных сторон.
2$2$.

\item
Периметр пятиконечной звезды равен $a$, периметр выпуклого пятиугольника
с~вершинами в~вершинах звезды равен $b$, периметр внутреннего пятиугольника
звезды равен $c$.
Докажите, что $b + c < 2 a$.

\item
Докажите, что в~выпуклом пятиугольнике найдутся три диагонали, из~которых можно
составить треугольник.

\item
Докажите, что в~параллелограмме против большего угла лежит большая диагональ.

\item
Биссектриса угла при основании~$BC$ равнобедренного треугольника $ABC$
пересекает боковую сторону~$AC$ в~точке~$K$.
Докажите, что $BK < 2 CK$.

\item
Отрезки $AB$ и~$CD$ длины~1 пересекаются в~точке~$O$, причем
$\angle AOC = 60^\circ$.
Докажите, что $AC + BD \geq 1$.

\item
\sp
Даны два выпуклых многоугольника $P$ и~$Q$, причем $P$ находится строго
внутри $Q$.
Докажите, что периметр $P$ меньше периметра $Q$.
\\
\sp
Верно~ли это для невыпуклых многоугольников?

\end{problems}

