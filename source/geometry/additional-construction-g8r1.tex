% $date: 2015-06-10
% $timetable:
%   g8r1:
%     2015-06-10:
%       2:

\section*{Дополнительные построения}

% $authors:
% - Илья Богданов

\begin{problems}

\item
В~трапеции $ABCD$ биссектриса угла~$D$ делит боковую сторону~$AB$ пополам.
Докажите, что $AD + BC = CD$.

\item
Треугольник $ABC$ таков, что $\angle A = 40^\circ$, $\angle B = 20^\circ$
и~$AB - BC = 4$.
Найдите длину биссектрисы угла~$C$.

\item
Постройте трапецию $ABCD$ по~основаниям~$BC = b$ и~$AD = a$ ($a < b$)
и~диагоналям $AC = d_1$ и~$BD = d_2$.

\item
Медиана~$AM$ треугольника $ABC$ в~четыре раза меньше стороны~$AB$
и~$\angle BAM = 60^\circ$.
Найдите угол $\angle BAC$.

\item
На~сторонах $BC$ и~$CD$ квадрата $ABCD$ взяты соответственно точки $M$ и~$N$
так, что $\angle MAN = 45^\circ$.
$AH$~--- высота треугольника $AMN$.
Докажите, что $AH = AB$.

\item
В~треугольнике $ABC$ проведена биссектриса~$AE$.
Оказалось, что $AE = EC$ и~$2 AB = AC$.
Найдите углы треугольника.

%\item
%Даны правильные треугольники $ABC$ и~$ADF$.
%Известно, что точка~$D$ расположена на~стороне~$BC$ так, что отрезки~$DF$
%и~$AB$ пересекаются.
%Кроме того, на~стороне~$BC$ отмечена такая точка~$E$, что $BD = EC$.
%Докажите, что $AB$ перпендикулярно $EF$.

\item
Отрезки $AB$ и~$CD$ длины~1 пересекаются в~точке~$O$, причем
$\angle AOC = 60^\circ$.
Докажите, что $AC + BD \geq 1$.

\item
В~равнобедренном $AB = AC$ треугольнике $ABC$ биссектриса угла~$A$ вдвое
короче биссектрисы угла~$B$.
Найдите углы треугольника $ABC$.

%\item
%В~трапеции сумма углов при одном из~оснований равна $90^\circ$.
%Найдите длину отрезка, соединяющего середины оснований, если длина отрезка,
%соединяющего середины диагоналей, равна $d$.

\end{problems}

