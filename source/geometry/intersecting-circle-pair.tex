% $date: 2015-06-03
% $timetable:
%   g9r3:
%     2015-06-04:
%       1:
%   g9r2:
%     2015-06-03:
%       2:
%   g9r1:
%     2015-06-03:
%       1:

\section*{Две пересекающиеся окружности}

% $authors:
% - Александр Блинков

\begin{problems}

\item
Две окружности пересекаются в~точках $P$ и~$Q$.
Через точку $P$ проведена секущая, которая пересекает окружности
в~точках $A$ и~$B$, а~через точку~$Q$~--- секущая, которая пересекает
окружности в~точках $C$ и~$D$ соответственно.
Докажите, что:
\\
\subproblem $\angle AQB = \angle CPD$;
\quad
\subproblem $AC \parallel BD$;
\\
\subproblem $AC = BD$ тогда и~только тогда, когда $AB \parallel CD$.

\item
Две окружности пересекаются в~точках $A$ и~$B$.
Через точку~$A$ проведена секущая, которая пересекает окружности
в~точках $C$ и~$D$.
Через точки $C$ и~$D$ проведены касательные к~окружностям, которые пересекаются
в~точке~$E$.
Докажите, что:
\\
\subproblem четырехугольник $BCED$~--- вписанный;
\\
\subproblem величина $\angle CED$ не~зависит от~выбора секущей.

\item
На~хорде~$AB$ окружности с~центром~$O$ выбрана произвольная точка~$C$.
Через точки~$A$, $O$ и~$C$ проведена окружность, которая пересекает данную
окружность в~точке~$D$.
Докажите, что треугольник $BCD$~--- равнобедренный.

\item
Две окружности пересекаются в~точках $P$ и~$Q$.
Прямая пересекает эти окружности последовательно в~точках $A$, $B$, $C$ и~$D$.
Докажите, что $\angle APB = \angle CQD$.

\item
Две окружности пересекаются в~точках $P$ и~$Q$.
Через точку~$P$ проведена секущая, которая пересекает окружности
в~точках $A$ и~$B$.
Найдите геометрическое место центров окружностей, описанных около
треугольника $AQB$.

\item
Две окружности пересекаются в~точках $A$ и~$B$.
Через произвольную точку~$X$ первой окружности проведена прямая~$XA$, которая
пересекает вторую окружность в~точке~$Y$ и~прямая~$XB$, которая пересекает
вторую окружность в~точке~$Z$.
Докажите, что:
\\
\subproblem
прямая~$YZ$ перпендикулярна диаметру первой окружности, проведенному через
точку~$X$;
\\
\subproblem
высоты всех таких треугольников $XYZ$, проведенные из~точки~$X$, пересекаются
в~одной точке;
\\
\subproblem
биссектрисы всех таких треугольников $XYZ$, проведенные из~точки~$X$,
пересекаются в~одной точке.

\item
Две окружности пересекаются в~точках $P$ и~$Q$.
Общая касательная к~этим окружностям касается их~в~точках $A$ и~$B$.
Докажите, что:
\\
\subproblem
прямая~$PQ$ пересекает отрезок~$AB$ в~его середине;
\\
\subproblem
равны радиусы окружностей, описанных около треугольников $APB$ и~$AQB$;
\\
\subproblem
окружность, описанная около одного из~этих треугольников, проходит через
ортоцентр другого треугольника.

\end{problems}

