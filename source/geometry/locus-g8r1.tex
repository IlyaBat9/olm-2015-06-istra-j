% $date: 2015-06-05
% $timetable:
%   g8r1:
%     2015-06-05:
%       2:

\section*{Геометрическое место точек}

% $authors:
% - Андрей Меньщиков

\definition
\emph{Геометрическое место точек (ГМТ)}~--- это множество точек плоскости,
которые удовлетворяют указанному свойству или соотношению.

\claim{Простейшие примеры}
\\
ГМТ, равноудаленных от концов отрезка,~--- серединный перпендикуляр к этому
отрезку;
\\
ГМТ, равноудаленных от лучей данного угла,~--- биссектриса этого угла;
\\
ГМТ, из которых данный отрезок виден под прямым углом,~--- окружность без двух
точек;
\\
ГМТ, удаленных от данной точки на расстояние $R > 0$,~--- окружность.

\claim{Упражнение 1}
Даны точки $A$, $B$, $C$, не лежащие на одной прямой.
Найдите ГМТ $M$ таких, что
\\
\subproblem
прямая~$CM$ не пересекает отрезок~$AB$;
\quad
\subproblem
отрезок~$CM$ пересекает отрезок~$AB$.

\claim{Упражнение 2}
Докажите, что серединные перпендикуляры к сторонам треугольника пересекаются
в одной точке, \emph{центре описанной окружности}.

\begin{problems}

\item
Дан отрезок~$AB$.
Найдите геометрическое место точек~$C$ таких, что
\\
\sp $\angle{BAC} = 17^{\circ}$;
\quad
\sp $AC < BC$;
\quad
\sp $AC + BC = AB$;
\\
\sp высота~$CH$ треугольника $ABC$ равна $h$;
\\
\sp медиана~$CM$ треугольника $ABC$ равна $m$;
\\
\sp треугольник $ABC$~--- равнобедренный;
\\
\sp треугольник $ABC$~--- остроугольный;
\\
\sp треугольник $ABC$~--- тупоугольный;
\\
\sp
расстояние от $C$ до какой-нибудь из двух точек $A$ и $B$ меньше длины
отрезка~$AB$.
\\
\sp
$\angle B $~--- второй по величине угол треугольника $ABC$
(если два меньших угла или все три угла треугольника равны, то вторым
по величине считается каждый из них).

%\item
%Даны две параллельные прямые.
%Найдите ГМТ, для которых расстояние до первой вдвое больше, чем до второй.

%\item
%Дан квадрат $ABCD$.
%Найдите ГМТ, которые расположены ближе к центру квадрата, чем к любой его
%вершине.

\item
Дан квадрат $ABCD$.
Найдите ГМТ таких, что сумма расстояний от каждой из них до прямых $AB$ и $CD$
равна сумме расстояний до прямых $BC$ и $AD$.

\item
Найдите ГМТ внутренних точек прямоугольника $ABCD$ таких, что
$S_{ABM} + S_{CDM} = S_{ADM} + S_{BCM}$.

\item
\sp
Найдите ГМТ, равноудаленных от двух данных прямых.
\\
\sp
Докажите, что три биссектрисы треугольника пересекаются в одной точке,
\emph{центре вписанной окружности}.
\\
\sp
Докажите, что биссектриса внутреннего угла и две биссектрисы внешних углов
треугольника пересекаются в одной точке, \emph{центре вневписанной окружности}.

\item
\sp
Дан треугольник $ABC$.
Найдите ГМТ $X$ таких, что $S_{ABX} = S_{CBX}$.
\\
\sp
Докажите, что медианы треугольника пересекаются в одной точке.

\item
\sp
Дан треугольник $ABC$.
Найдите ГМТ $X$ таких, что $AX^2 - XC^2 = AB^2 - BC^2$.
\\
\sp
Докажите, что высоты треугольника пересекаются в одной точке.

\item
Лестница стоит вплотную к стене, а к ее середине прилипла муха.
Лестница плавно скользит, и в итоге падает на пол
(ее верхний конец всегда прижат к стене).
Какова траектория мухи?

\item
Для выпуклого четырехугольника $ABCD$ верно $AB < BC$ и $AD < DC$.
Точка~$M$ лежит на диагонали~$BD$.
Докажите, что $AM < MC$.

\end{problems}

