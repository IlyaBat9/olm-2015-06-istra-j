% $date: 2015-06-12
% $timetable:
%   g8r1:
%     2015-06-12:
%       2:

\section*{Геометрический разнобой}

% $authors:
% - Андрей Меньщиков

\begin{problems}

\item
Существует~ли такой треугольник, что
\\
\sp
все его стороны больше $1\,\text{км}$, а~площадь меньше $1\,\text{см}^2$?
\\
\sp
все его высоты меньше $1\,\text{см}$, а~площадь больше $1\,\text{км}^2$?
\\
\sp
все его стороны меньше $1\,\text{см}$, а~площадь больше $1\,\text{см}^2$?

\item
На~сторонах $BC$ и~$CD$ квадрата $ABCD$ отмечены точки $M$ и~$K$ соответственно
так, что $\angle BAM = \angle CKM = 30^{\circ}$.
Найдите $\angle AKD$.

\item
Пусть $B B_1$~--- биссектриса неравнобедренного треугольника $ABC$ с~углом
$\angle B = 38^{\circ}$.
Из~точки~$O$, лежащей на~луче~$B B_1$, опустили перпендикуляр~$OH$
на~сторону~$AC$.
Оказалось, что $AH = HC$.
Найдите $\angle OAC$.

%\item
%На~гипотенузе~$AB$ прямоугольного треугольника $ABC$ во~внешнюю сторону
%построен квадрат с~центром в~точке~$O$.
%Докажите, что $CO$~--- биссектриса прямого угла~$C$.

\item
В~треугольнике $ABC$ стороны $AB$ и~$BC$ равны.
На~прямой~$AC$ выбрана точка~$D$ такая, что $A$~--- середина~$DC$.
Перпендикуляр к~прямой~$DC$ в~точке~$A$ пересекает отрезок~$BD$ в~точке~$E$.
Докажите, что $\angle DBA = \angle BCE.$

\item
Найдите геометрическое место внутренних точек данного угла, сумма расстояний
от~которых до~сторон данного угла равна заданной величине.

\item
Точка~$F$~--- середина медианы~$BD$ треугольника $ABC$.
Точка~$E$ на~стороне~$BC$ такова, что $DE \perp BC$.
Докажите, что если $AB = AE$, то~$\angle AFD = \angle FED$.

\item
Даны правильные треугольники $ABC$ и~$ADF$.
Известно, что точка~$D$ расположена на~стороне~$BC$ так, что отрезки~$DF$
и~$AB$ пересекаются.
Кроме того, на~стороне~$BC$ отмечена такая точка~$E$, что $BD = EC$.
Докажите, что $AB$ перпендикулярно $EF$.

\item
На~стороне~$AC$ треугольника $ABC$ отмечена точка~$D$ такая, что
$AC = BD$ и~$\angle ABD = 25^{\circ}$.
Известно также, что $\angle BAC = 40^{\circ}$.
Докажите, что $AD + BC > AB$.

\item
В~вершине~$A$ единичного квадрата $ABCD$ сидит муравей.
Ему надо добраться до~точки~$C$, где находится вход в~муравейник.
Точки $A$ и~$C$ разделяет вертикальная стена, имеющая вид равнобедренного
прямоугольного треугольника с~гипотенузой~$DB$.
Найдите длину кратчайшего пути, который надо преодолеть муравью, чтобы попасть
в~муравейник.

\end{problems}

