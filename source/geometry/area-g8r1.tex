% $date: 2015-06-03
% $timetable:
%   g8r1:
%     2015-06-03:
%       2:

\section*{Площади}

% $authors:
% - Андрей Меньщиков

\definition
Каждой фигуре~$M$ на плоскости мы сопоставим число~$S_M$, называемое
\emph{площадью}, обладающее следующими свойствами:
\\
\textbf{(1)}\enspace
Площадь неотрицательна;
\\
\textbf{(2)}\enspace
Площади равных фигур равны;
\\
\textbf{(3)}\enspace
Площадь объединения фигур, не имеющих общих внутренних точек, есть сумма
площадей фигур;
\\
\textbf{(4)}\enspace
Площадь прямоугольника равна произведению его сторон.

\begin{problems}

\item
Дан прямоугольник $ABCD$.
\\
\sp
На $BC$ взята точка~$K$.
Докажите, что $S_{ABCD} = 2 S_{AKD}$.
\\
\sp
На $BC$ взяты точка $K$ и $L$.
Докажите, что $S_{AKD} = S_{ALD}$.
\\
\sp
На прямой~$BC$ взята точка~$K$, а на прямой~$AD$ взята точка~$L$.
Докажите, что $S_{AKD} = S_{BLC}$.

\end{problems}

\claim{Факт}
Даны две параллельные прямые $AB$ и $l$.
Тогда площадь $\triangle ABC$ не зависит от от выбора точки~$C$, находящейся
на прямой $l$.

\begin{problems}

\item
В трапеции $ABCD$ с меньшим основанием~$BC$ через точку~$B$ проведена прямая,
параллельная $CD$ и пересекающая диагональ~$AC$ в точке~$E$.
Сравните площади $\bigtriangleup ABC$ и $\bigtriangleup DEC$.

\item
Докажите, что в трапеции с проведенными диагоналями треугольники, прилежащие
к боковым сторонам, равновелики.

%\item
%$ABCD$~--- трапеция, $MN$~--- ее средняя линия.
%Докажите, что площади $\triangle MCN$ и $\triangle AMN$ равны.

\item
Через точку~$D$, лежащую на стороне~$BC$ треугольника $ABC$, проведены прямые,
параллельные двум другим сторонам и пересекающие $AB$ и $AC$ соответственно
в точках $E$ и $F$.
Докажите, что $S_{CDE} = S_{BDF}$.

\item
Точка~$E$~--- середина стороны~$BC$ треугольника $ABC$.
Точки $D$ и $G$ на сторонах $BC$ и $AC$ соответственно таковы, что
$AD \perp BC$ и $GE \perp BC$.
Докажите, что $S_{CGD} = S_{DGAB}$.

\item
Докажите, что медианы разбивают треугольник на шесть равновеликих
треугольников.

\item
Пол в комнате площадью~$S$ покрыт линолеумом общей площадью~$S$ так, что нет
участков, покрытых более чем в два слоя.
Докажите, что площадь пола, покрытая дважды, равна площади пола, не покрытой
ни разу.

\item
\label{geometry/area-g8r1:problem:find area}%
Рис.~\ref{geometry/area-g8r1:problem:find area:fig}:
выразите $X$ через $S$.
Там, где не указано, $X = S_{ABC}$.

\begin{figure}[ht]
\begin{center}
    \hfill
    \jeolmfigure[height=0.2\linewidth]{area-1}
    \hfill
    \jeolmfigure[height=0.2\linewidth]{area-2}
    \hfill
    \jeolmfigure[height=0.2\linewidth]{area-3}
    \hfill
    \jeolmfigure[height=0.2\linewidth]{area-4}
    \hfill
    \caption{к задаче~\ref{geometry/area-g8r1:problem:find area}}
    \label{geometry/area-g8r1:problem:find area:fig}
\end{center}
\end{figure}

\item
Диагонали выпуклого четырехугольника делят его на 4 части, площади которых,
взятые последовательно, равны $S_1$, $S_2$, $S_3$, $S_4$.
Докажите, что $S_1 \cdot S_3 = S_2 \cdot S_4$.

\item
Внутри правильного треугольника выбрали точку.
Докажите, что сумма расстояний от нее до сторон треугольника от данного выбора
не зависит.
А верно ли это для правильного пятиугольника?

\item
Каждая из сторон выпуклого четырехугольника разделена на три равные части,
и соответствующие точки противоположных сторон соединены.
Докажите, что площадь центрального четырехугольника в девять раз меньше площади
целого.

\end{problems}

