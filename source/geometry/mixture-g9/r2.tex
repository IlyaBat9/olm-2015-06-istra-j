% $date: 2015-06-10
% $timetable:
%   g9r2:
%     2015-06-10:
%       1:
%     2015-06-12:
%       1:

\section*{Разнобой-повторение}

% $authors:
% - Фёдор Львович Бахарев

\begin{problems}

\item
Четырехугольник $ABCD$ вписан в~окружность.
Пусть $H_A$, $H_B$, $H_C$, $H_D$~--- ортоцентры треугольников
$BCD$, $CDA$, $DAB$ и~$ABC$ соответственно.
Докажите, что
\\
\subproblem $H_A H_B = AB$;
\quad
\subproblem $H_A H_B H_C H_D$~--- четырехугольник, равный $ABCD$.

\item
Четырехугольник $ABCD$ вписан в~окружность.
Пусть $I_A$, $I_B$, $I_C$, $I_D$~---
центры вписанных окружностей треугольников $BCD$, $CDA$, $DAB$ и~$ABC$
соответственно.
Докажите, что $I_A I_B I_C I_D$~--- прямоугольник.

\item
На~окружности зафиксированы точки $A$ и~$B$.
Точка~$C$ движется по~дуге окружности.
\\
\subproblem
Докажите, что точка~$I_a$ пересечения биссектрис внешних углов $B$ и~$C$
треугольника $ABC$ движется по~дуге некоторой окружности.
\\
\subproblem
Укажите центр окружности, по~которой движется точка~$I_a$.

\item
Из~точки пересечения $A$ двух окружностей одновременно с~одинаковыми угловыми
скоростями стартовали два велосипедиста, оба против часовой стрелки.
Докажите, что середина отрезка, их~соединяющего, движется по~окружности.

\item
Дан треугольник $ABC$;
из~точек $A$ и~$C$ одновременно с~равными скоростями стартуют две мухи
в~сторону точки~$B$.
\\
\subproblem
Укажите точку, от~которой они постоянно равноудалены.
\\
\subproblem
Докажите, что середина отрезка между мухами движется по~прямой, параллельной
биссектрисе угла~$B$.

\item
Даны два квадрата $ABCD$ и~$AEFG$ с~общей вершиной~$A$, вершины указаны против
часовой стрелки.
Докажите, что прямые $BE$, $CF$ и~$DG$ пересекаются в~одной точке и~найдите
углы между этими прямыми.

\item
Докажите, что вершины $A$ и~$B$ остроугольного треугольника $ABC$,
ортоцентр~$H$, а~также проекция~$H$ на~медиану из~вершины~$C$ лежат на~одной
окружности.

\end{problems}

