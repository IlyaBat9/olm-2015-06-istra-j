% $date: 2015-06-10
% $timetable:
%   g9r1:
%     2015-06-10:
%       1:
%     2015-06-11:
%       2:

\section*{Разнобой-повторение}

% $authors:
% - Фёдор Львович Бахарев

\begin{problems}

\item
Четырехугольник $ABCD$ вписан в~окружность.
Пусть $H_A$, $H_B$, $H_C$, $H_D$~--- ортоцентры треугольников
$BCD$, $CDA$, $DAB$ и~$ABC$ соответственно.
Докажите, что $H_A H_B H_C H_D$~--- четырехугольник, равный $ABCD$.

\item
Четырехугольник $ABCD$ вписан в~окружность.
Пусть $I_A$, $I_B$, $I_C$, $I_D$~--- центры вписанных окружностей треугольников
$BCD$, $CDA$, $DAB$ и~$ABC$ соответственно.
Докажите, что $I_A I_B I_C I_D$~--- прямоугольник.

\item
На~окружности зафиксированы точки $A$ и~$B$.
Точка~$C$ движется по~дуге окружности.
\\
\sp
Докажите, что точка~$I_a$ пересечения биссектрис внешних углов $B$ и~$C$
треугольника $ABC$ движется по~дуге некоторой окружности.
\\
\sp
Укажите центр окружности, по~которой движется точка~$I_a$.

\item
Четырехугольник $ABCD$ вписан в~окружность.
Опишите взаимное расположение 16-ти центров вписанных и~вневписанных
окружностей треугольников $BCD$, $CDA$, $DAB$ и~$ABC$.

\item
Дан треугольник $ABC$;
из~точек $A$ и~$C$ одновременно с~равными скоростями стартуют две мухи
в~сторону точки~$B$.
\\
\sp
Укажите точку, от~которой они постоянно равноудалены.
\\
\sp
Докажите, что середина отрезка между мухами движется по~прямой, параллельной
биссектрисе угла~$B$.

\item
Даны два правильных семиугольника с~общей вершиной.
Вершины каждого семиугольника нумеруются цифрами от~$1$ до~$7$ по~часовой
стрелке, причем в~общей вершине ставится~1.
Вершины с~одинаковыми номерами соединены прямыми.
Докажите, что полученные шесть прямых пересекаются в~одной точке.

\item
$ABCD$~--- вписанный четырехугольник.
$M$~--- точка пересечения его диагоналей.
Некоторая прямая, проходящая через точку~$M$, пересекает
окружность, описанную около $ABCD$, в~точках $M_1$ и~$M_2$, и~окружности,
описанные около треугольников $ABM$ и~$CDM$, в~точках $N_1$ и~$N_2$.
Докажите, что $M_1 N_1 = M_2 N_2$.

\end{problems}

