% $date: 2015-06-06
% $timetable:
%   g9r3:
%     2015-06-07:
%       2:

\section*{Передвижение точек}

% $authors:
% - Фёдор Бахарев

\begin{problems}

\item
В~середине лестницы, приставленной к~стене сидит котенок.
Лестница начинает сползать по~стене (и~скользить по~полу).
Какую траекторию описывает котенок?

\item
Две окружности пересекаются в~точках $A$ и~$B$.
Из~точки $A$ одновременно стартуют два велосипедиста.
Каждый из~них едет по~своей окружности против часовой стрелки.
При этом угловые скорости у~них одинаковые.
Докажите, что прямая, соединяющая их, все время проходит через точку~$B$.

\item
На~окружности зафиксированы точки $A$ и~$B$.
Точка~$C$ движется по~дуге окружности.
Докажите, что точка пересечения $H$ высот треугольника $ABC$ движется
по~окружности, симметричной описанной окружности треугольника $ABC$
относительно стороны~$AB$, если
\\
\sp треугольник $ABC$ остроугольный;
\\
\sp если угол $C$ тупой;
\\
\sp если угол $A$ тупой.

\item
На~окружности зафиксированы точки $A$ и~$B$.
Точка~$C$ движется по~дуге окружности.
\\
\sp
Докажите, что точка~$I$ пересечения биссектрис треугольника $ABC$ тоже движется
по~дуге некоторой окружности.
\\
\sp
Укажите центр окружности, по~которой движется точка~$I$.

\item
Две мухи ползут по~сторонам угла с~вершиной~$O$ с~одинаковой постоянной
скоростью, одна в~сторону точки~$O$, вторая~--- от~точки~$O$.
В~какой-то момент они оказались на~одинаковом расстоянии
от~некоторой точки~$A$ на~биссектрисе угла.
Докажите, что они либо все время от~нее равноудалены, либо этого больше никогда
не~произойдет.

\item
Велосипедисты едут по~двум концентрическим окружностям с~центром~$O$
с~одинаковыми угловыми скоростями против часовой стрелки.
В~начальный момент их~положения $A_1$ и~$B_1$ таковы, что
$O A_1 \perp A_1 B_1$.
Докажите, что если в~какой-то момент они располагаются в~точках $A_2$ и~$B_2$,
то~$A_1 A_2$ делит $B_1 B_2$ пополам.

\item
Две окружности с~центрами $O_1$ и~$O_2$ пересекаются в~точках $A$ и~$B$.
Из~точки~$A$ одновременно стартуют два велосипедиста.
Каждый из~них едет по~своей окружности один по~часовой стрелке, другой~---
против.
Угловые скорости у~них одинаковые.
Четырехугольник $A O_1 C O_2$~--- параллелограмм.
Докажите, что велосипедисты постоянно равноудалены от~точки~$C$.

\end{problems}

